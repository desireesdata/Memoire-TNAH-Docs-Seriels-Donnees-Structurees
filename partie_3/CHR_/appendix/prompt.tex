\section{Full prompt} \label{appdx:full-prompt}
This appendix features an English translation of the complete prompt used in our experiment, visible in \Cref{fig:full_prompt}.
\begin{figure}[!h]
\begin{minipage}{\textwidth}
\small
\begin{verbatim}
<TASK TO DO>: Extract from the text I am about to give you,
information from each entry, each of which relates to one person.
<NEED TO KNOW>: First of all, be aware that there is one entry per
person and that, for context, the people mentioned have participated
in the activity of the Senate. They are generally senators, ministers,
undersecretaries, etc.
<ENTRIES>: Each entry consists of: the NAME and sometimes the FIRST
NAME of a speaker (str); sometimes his role (this is not always
specified); a list of ACTIONS he has performed or which concern him.
<ACTION>: Each action concerning a speaker is generally linked to one
or more page numbers. When there is a page reference, you can be sure
that it is a reference to an action concerning the stakeholder.
<INDEX REFERENCE CASE>: In the case where an entry does NOT set out
actions or facts and/or pages concerning a speaker, but a simple
nominal mention, then it is an index reference. In this case, you
should indicate the reference of the reference (str). These references
are generally the first names and surnames of contributors. These
references therefore do not refer to pages, but to other nominal
entries.
<HERE IS THE INFORMATION TO BE EXTRACTED>: So I want you to give me
the surnames (and first names if there are any); as well as the page
numbers relating to the descriptions of the actions or interventions
of each speaker -- List[int]-- OR, if there is no action, just say
that it is an index reference ("<index reference>") -- (str).
<NOTE>: - When there is no index reference (a str), adopt this syntax
at the appropriate level: "references_pages":"<index_reference>".
- First names must be placed in brackets and after the name.
- If a page reference appears several times in an entry (i.e. for the
same speaker), there is no need to repeat it.
<ATTENTION>: The text submitted to you may be truncated. If this is
the case, ignore the incomplete text and consider only the complete
entries.
SO HERE IS THE TEXT from which you need to extract the information:
\end{verbatim}
\end{minipage}
\caption{Submitting a request to an LLM via a the Mistral API, including entity extraction instructions, the raw text, and the expected data schema. }
\label{fig:full_prompt}
\end{figure}
