\section{Corpus Details}
\label{appdx:corpora-details}
% !!!!!!!!!!!!!!!!!!!!!!!
% extracted from section 3, not necessarily to be included in the final version
% !!!!!!!!!!!!!!!!!!!!!!!


The corpus under study comprises digitized serial documents from the Journal Officiel, available via the Gallica platform: specifically, the 1931 Tables nominatives—or "Tables des noms".

The Tables nominatives are part of the \textit{Tables du Journal Officiel}, which index the editions of the \textit{Journal Officiel} including, notably, the \textit{Lois et décrets} edition, and the two \textit{Débats parlementaires} editions -- one for the Senate and another for the Chamber of Deputies. Thus, within the \textit{Journal Officiel}'s documentary ecosystem, which aims to reconstruct parliamentary activity and its legal or regulatory outcomes in France, the Senate's \textit{Tables nominatives} provide a concise record of senators' activity in public session. These interventions, referenced and briefly described in the tables, are therefore detailed in the \textit{Débats} editions. These "\textit{in extenso}" transcriptions of speeches are carried out by a dedicated stenographic service that reviews and harmonizes interventions made in public session, in collaboration with the services of the \textit{Direction des Journaux Officiels}, which is attached to the Ministry of the Interior.

There's an edition of the \textit{Tables du Journal Officiel} for each year, typically comprising around 450 pages in the 1930s. The \textit{Tables des noms} section—including both Senate and Chamber of Deputies -- spans about forty pages, with the Senate portion generally covering around fifteen pages. For the year 1931, the Senate's \textit{Tables des noms} contains 14 pages and approximately 300 entries, representing as many interventions. Each entry corresponds to a participant, detailing their various actions (requests for interpellation, discussions of bills, reading of commission opinions, submission of amendments, etc.) and providing the page reference that leads to the in extenso transcription of the intervention. These transcriptions are published in the Senate's \textit{Débats parlementaires} edition. Thus, the \textit{Tables} are linked to the intervention transcriptions via page references. Furthermore, since pagination is continuous throughout the year, the exact date of a parliamentary intervention can be precisely determined from its page reference.

The objective, therefore, via structured output, is to extract information related to parliamentary activity in the Senate for the year 1931. 

The digitized index images were retrieved via Gallica'’'s IIIF image API\footnote{More information on the API can be found here: \url{https://api.bnf.fr/fr/api-iiif-de-recuperation-des-images-de-gallica}.}.
