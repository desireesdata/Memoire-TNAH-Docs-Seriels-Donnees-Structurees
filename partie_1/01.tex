

\section{Le Journal Officiel : la parole, la main, la vue et le droit}

qsd

\subsection{Les sources sérielles : définition et enjeux}

Les sources sérielles désignent des ensembles documentaires produits de manière régulière, cumulative et normalisée, qui permettent d’observer la répétition et l’évolution de phénomènes sociaux ou politiques dans le temps. Pierre Chaunu, qui en a popularisé l’usage dans les années 1960, les définissait comme des matériaux « à série longue », permettant une histoire quantitative de la conjoncture, qu’il s’agisse des prix, des registres paroissiaux ou des trafics maritimes. Dans la lignée de cette approche, Emmanuel Le Roy Ladurie a montré que la régularité et la masse de ces sources ouvraient la voie à une « histoire sérielle » fondée sur l’accumulation et le traitement statistique.

À l’ère numérique, des historiens comme Frédéric Clavert rappellent que ces sources se distinguent moins par leur nature que par leur mode de production : leur caractère répétitif, leur homogénéité formelle et leur relative standardisation les rendent propices à des opérations d’extraction et de structuration automatisées. En ce sens, elles ne livrent pas seulement des contenus, mais offrent un potentiel méthodologique, qui repose sur leur caractère cumulatif.

La sérialité documentaire ne doit cependant pas masquer la dimension éditoriale et institutionnelle de ces objets. Comme l’a montré Alain Desrosières pour les statistiques publiques, la constitution d’une série est toujours le résultat d’un choix social et politique : décider ce qui est compté, comment cela est classé, et pour quels usages. Appliqué aux sources parlementaires, ce constat invite à considérer que les tables du Journal Officiel ne sont pas de simples instruments techniques : elles orientent la lecture, hiérarchisent les thèmes et construisent une représentation ordonnée de l’activité politique.

\subsection{1.1. Le Journal Officiel : entre publicité et promulgation}

La création du \emph{Journal Officiel de la République française} en 1869 s’inscrit dans une longue histoire des dispositifs de publicité de la loi. Héritier à la fois du \emph{Moniteur universel} (1789–1869), qui transcrivait les débats parlementaires, et du \emph{Bulletin des lois} (1793–1931), garant de la promulgation exécutoire des textes normatifs, le \emph{Journal Officiel} cumule deux fonctions : informer le public des débats parlementaires et conférer force obligatoire aux lois par leur publication. Cette double vocation, attestée dès la Révolution française, fait de ce périodique un instrument essentiel de la transparence parlementaire et de l’effectivité juridique.

Son caractère sériel — une parution quasi quotidienne, au format stable, avec une organisation régulière des rubriques — constitue précisément ce qui en fait une source de premier plan pour l’historien. Sa régularité et sa cumulativité permettent de suivre à la trace l’activité parlementaire, mais elles posent aussi la question de la matérialité du texte publié : transcription fidèle ou recomposition éditoriale ? Dès lors, le \emph{Journal Officiel} ne doit pas être lu seulement comme un réceptacle de la parole parlementaire, mais comme une œuvre éditoriale, façonnée par les sténographes, rédacteurs et correcteurs qui traduisent l’oralité en texte écrit.

\subsection{1.2. Archives ou documentation ? La place singulière du \emph{Journal Officiel}}

La nature archivistique du \emph{Journal Officiel} demeure ambivalente. Juridiquement, il s’agit d’une publication soumise au dépôt légal, conservée à ce titre à la Bibliothèque nationale de France, à la bibliothèque des Assemblées et aux Archives nationales. Pourtant, son intégration dans la \textbf{série K des Archives départementales} interroge : peut-on considérer comme archives des documents produits à des fins éditoriales, vendus au public et destinés à être lus ?

L’analyse archivistique rappelle que les archives sont le « collatéral direct » d’une activité administrative, produites sans intention de publication. Or, le \emph{Journal Officiel} est un objet éditorial dont la finalité première est précisément la publicité. Son inscription en série K — aux côtés des lois, ordonnances et arrêtés préfectoraux — illustre néanmoins la manière dont l’État a voulu garantir, dans chaque département, l’accès des citoyens aux textes normatifs. Loin d’être un simple effet du « respect des fonds », cette intégration témoigne d’une stratégie politique de diffusion centralisée du droit.

Cette tension entre document édité et document d’archives souligne la spécificité des sources sérielles : elles se situent à la frontière entre mémoire administrative et communication publique, entre traces institutionnelles et dispositifs de légitimation.

\subsection{1.3. Le Journal Officiel dans son contexte technique et administratif (1921–1940)}

Pour comprendre la matérialité de la source exploitée dans ce mémoire (les tables du Sénat de 1931), il faut rappeler que le \emph{Journal Officiel} n’était pas seulement le produit d’une Chambre parlementaire, mais celui d’une organisation industrielle. Depuis 1880, la publication est assurée par la \textbf{Société anonyme coopérative de composition et d’impression des Journaux officiels (SACIJO)}, placée sous tutelle du ministère de l’Intérieur. Linotypistes, rotativistes, correcteurs et personnels administratifs concourent à sa fabrication quotidienne.

La période 1921–1940, bornée par l’acquisition de la \emph{linotype Model 9} et l’interruption de 1940, offre un cadre technique relativement homogène. Elle correspond aussi à une stabilité des pratiques éditoriales, qui permet d’envisager une lecture sérielle. Les archives de la Direction des Journaux officiels (série 19840069 des AN) éclairent cette fabrique administrative : rapports budgétaires, organigrammes, correspondances de service. Elles révèlent un fonctionnement hybride, entre administration d’État et entreprise de presse, qui explique la diffusion massive et régulière de ces volumes.

\subsection{1.4. Les « processus métier » de la publicité parlementaire}

Qualifier la chaîne de production parlementaire en termes de « processus métier » revient à cartographier l’ensemble des opérations qui transforment la parole politique en texte normatif publié. À la IIIe République, ce processus suit plusieurs étapes :

* \textbf{Délibérer} : débats oraux au Sénat et à la Chambre des députés, régis par les règlements de 1876, avec une organisation en bureaux et commissions.
\emph{ \textbf{Transcrire} : sténographes et rédacteurs produisent les comptes rendus }in extenso*, qui passent par un travail de révision avant impression.
\emph{ \textbf{Publier} : les textes sont édités dans le }Journal Officiel* et diffusés par abonnement et dépôt légal.
* \textbf{Promulguer} : la publication confère force obligatoire aux lois votées, qui ne prennent effet qu’une fois rendues publiques.

Ce continuum — délibérer, transcrire, publier, promulguer — rend manifeste le rôle central du \emph{Journal Officiel}. Il ne s’agit pas seulement d’un témoin documentaire, mais d’un maillon de l’effectivité du droit.

\chapter{Chapitre 2 — Les tables annuelles : des relations entre corpus}

\subsection{2.1. Les tables dans l’environnement du \emph{Journal Officiel}}

À côté des livraisons quotidiennes du \emph{Journal Officiel}, le dispositif documentaire de la Troisième République produit un ensemble d’outils de repérage et de cumul : index, tables et recueils annuels. Ces tables, organisées par Chambre et par type de document (séances, questions, interventions, lois, décrets, etc.), constituent un instrument de navigation à travers la masse documentaire accumulée. Elles offrent un second niveau de structuration, indispensable à l’exploitation d’un corpus qui, sans cela, serait pratiquement illisible dans son entier.

Dans ce sens, les tables ne sont pas de simples annexes, mais un élément constitutif du \emph{Journal Officiel}. Leur publication témoigne d’une volonté de rendre praticable la lecture sérielle, en transformant un flot continu de débats en une matière consultable a posteriori. Elles permettent aux parlementaires, aux fonctionnaires et aux juristes, mais aussi aux journalistes et au public, de retrouver un débat, une loi ou un orateur dans un ensemble potentiellement infini de pages.

\subsection{2.2. Forme et organisation des tables}

La table annuelle se présente comme un volume imprimé, distinct des numéros quotidiens mais reprenant la même logique typographique de sobriété. La structuration est généralement alphabétique ou thématique, avec des entrées renvoyant à des numéros de séance ou de page du \emph{Journal Officiel}. Ainsi, le chercheur y trouve à la fois :

* des index de noms (parlementaires, ministres, orateurs) ;
* des index de matières (projets de lois, sujets débattus, thèmes abordés) ;
* des références législatives (dates, intitulés, numéros de lois et décrets).

Cette composition apparemment simple reflète un travail complexe de collecte et de mise en ordre, qui engage des méthodes d’indexation encore largement manuelles dans les années 1930. Les tables matérialisent donc une double médiation : celle de la transcription sténographique, puis celle de la mise en indexation.

\subsection{2.3. Informations sémantiques et usages}

Les tables ne livrent pas seulement des renvois. Leur organisation alphabétique ou thématique suggère déjà une lecture orientée du corpus. En réordonnant les débats selon les sujets ou les personnes, elles produisent une représentation « secondaire » de l’activité parlementaire :

* \textbf{Pour l’historien}, elles permettent de cartographier les thèmes récurrents, d’identifier des trajectoires individuelles de parlementaires, ou encore de suivre la maturation d’une question dans le temps long.
* \textbf{Pour les juristes}, elles assurent un repérage efficace des textes normatifs, condition de la sécurité juridique.
* \textbf{Pour l’administration}, elles facilitent la réutilisation interne des débats et la circulation de l’information entre services.

En ce sens, les tables possèdent une valeur sémantique propre : elles ne sont pas de simples index, mais des instruments de catégorisation, qui hiérarchisent les contenus du \emph{Journal Officiel} et leur confèrent une visibilité inégale.

\subsection{2.4. Les tables comme « hub » intercorpus}

Enfin, les tables établissent des liens entre différents ensembles documentaires. Elles ne se limitent pas aux seuls débats parlementaires, mais relient ceux-ci aux autres publications officielles et à des corpus complémentaires. Elles servent d’articulation entre :

\emph{ les volumes quotidiens du }Journal Officiel* ;
\emph{ les recueils législatifs et réglementaires (par exemple le }Bulletin des lois*) ;
* les instruments internes des Chambres (procès-verbaux, rapports de commissions) ;
* les archives départementales (série K), où elles prennent place aux côtés d’autres formes de publicité administrative.

En occupant cette position nodale, les tables fonctionnent comme des « hubs documentaires » : elles permettent de passer d’un corpus à l’autre, et d’inscrire les débats dans l’écosystème plus large des pratiques de gouvernement.



\subsection{Exemple : Les Tables du Sénat, année 1931}

Le volume des \emph{Tables annuelles du Sénat} pour l’année 1931 se présente sous la forme d’un in-octavo relié, composé de plusieurs centaines de pages. La typographie, sobre et régulière, reprend les conventions du \emph{Journal Officiel} : colonnes étroites, numérotation continue, absence d’ornementation. L’ensemble se divise en sections distinctes, qui reflètent les usages concrets des lecteurs.

#### 1. Index des orateurs

On y trouve une \textbf{liste alphabétique des sénateurs}, chaque nom suivi de références aux séances où ils sont intervenus. Par exemple :

\begin{quote}\emph{Tardieu (André)} : interventions p. 312, 457, 892.\end{quote}

Cet index permet de retracer rapidement la présence et l’activité d’un parlementaire sur une année complète. Pour l’historien, il offre une base sérielle pour mesurer la visibilité des élus et la fréquence de leur participation aux débats.

#### 2. Table des matières thématiques

La deuxième section regroupe les débats par \textbf{matières} :

\emph{ }Finances publiques* : budget, impôts, emprunts.
\emph{ }Affaires étrangères* : traités, conventions, mandats.
\emph{ }Travail et questions sociales* : assurance chômage, législation ouvrière, retraites.

Chaque entrée renvoie à un numéro de séance du \emph{Journal Officiel}. Ce classement thématique reflète une logique documentaire propre, qui diffère de l’ordre chronologique des séances : il met en valeur la récurrence des thèmes et facilite leur repérage transversal.

#### 3. Références législatives et réglementaires

Enfin, les tables recensent les \textbf{lois votées et les décrets publiés} pendant l’année, assortis de leur date et de leur numéro. Ce registre, proche d’un répertoire législatif, assure le lien avec le \emph{Bulletin des lois} et, par extension, avec l’ensemble de la législation nationale.

\subsection{Analyse}

Cet exemple illustre trois dimensions essentielles des tables :

* Leur \textbf{fonction instrumentale} : elles servent avant tout de guide, destiné à faciliter la recherche d’une information précise dans un corpus immense.
* Leur \textbf{valeur sémantique} : en proposant une catégorisation (par personnes, thèmes, textes), elles produisent une image de l’activité parlementaire qui n’est pas neutre, mais orientée par le mode d’indexation.
* Leur \textbf{rôle intercorpus} : en mettant en relation débats, interventions et textes normatifs, elles constituent un point de jonction entre la parole parlementaire et le droit promulgué.
