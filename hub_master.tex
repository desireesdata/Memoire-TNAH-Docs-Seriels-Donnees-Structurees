%Document vide aux normes de l'École nationale des Chartes
%crée par J.B. Camps
%Dernières modifications E. Rouquette (03/2025)

%%%%%%%%%%%%%%%%%%%%%% PRÉAMBULE


%%%%%%%%%%%%%% partie obligatoire du préambule
\documentclass[12pt,twoside]{book}
\usepackage{fontspec}
\usepackage{xunicode}
\usepackage{polyglossia}
\usepackage{epigraph}
\usepackage{graphicx}
\graphicspath{
	{img/}             
	{partie_1/img/}
	{partie_2/img/}    
	{partie_3/img/}
	{sources/img/}
}
\usepackage{marginnote}
\setmainlanguage{french}%indiquer la langue principale du document
%\setotherlanguage{} %indiquer les autres langues utilisée


%%%%%%%%%%%%%%%%%%%%%%%%%%%%%%%%% PACKAGES UTILISÉS

\usepackage{csquotes} % les guillemets français
\usepackage{lettrine} %faire une lettrine (pas obligatoire)


\makeatletter
\def\blx@bbxfile@enc{biblatex-enc/enc.bbx}
\def\blx@cbxfile@enc{biblatex-enc/enc.cbx}
\makeatother
\usepackage[style=biblatex-enc/enc,sorting=nyt,maxbibnames=10]{biblatex}%charger le style de l'EnC (téléchargeable ici https://ctan.org/pkg/biblatex-enc)
\addbibresource{sources/memoire.bib} %le fichier bibliograhique. Exemple de chemin à partir du dossier où se trouve le document maître:Exemple ./dossierA/fichier.bib
%\defbibheading{}{\subsection*{}} Si l'on veut changer le titre de la/les bibliographie(s)


%%%Faire un ou plusieurs index

%\usepackage{imakeidx} %pour faire un ou plusieurs index
%\makeindex %commande pour générer l'index


%RAJOUTEZ ICI VOS PACKAGES
\usepackage{amsmath}
\usepackage[font=small]{caption} 
\usepackage{minted} 


%%%%%%%%%%%%%%%%%%%%%%%%%%%%%%%%% CONFIGURATION DE MISE EN PAGE

%%%%%% Les compteurs (sections, subsections, etc)


%%%%%% Les compteurs (sections, subsections, etc)
\renewcommand{\thesection}{\Roman{section}.}%On ne fait apparaître que le numéro de la section
\renewcommand{\thesubsection}{\arabic{subsection}.}%subsection en chiffres arabes
\renewcommand{\thesubsubsection}{\alph{subsubsection}.}%subsubsection en lettres minuscules
%Si l'on veut faire apparaître les subsubsection dans le table des matières (à commenter sinon)
\setcounter{tocdepth}{3}
\setcounter{secnumdepth}{3}  % La subsubsection (profondeur=3 dans la table des matières) apparait numérotée dans la TdM



%%%%%  Configurer le document selon les normes de l'école

\usepackage[margin=2.5cm]{geometry} %marges
\usepackage{setspace} % espacement qui permet ensuite de définir un interligne
\onehalfspacing % interligne de 1.5
\setlength\parindent{1cm} % indentation des paragraphes à 1 cm


%%%%% Mise en forme des headers (haut de page)

\usepackage{fancyhdr} %package utilisé pour modifier les headers
\pagestyle{fancy} %utiliser ses propres choix de mise en page et non ceux par défaut du package

\setlength\headheight{16pt}%la hauteur des headers

%%la façon dont les sections apparaissent dans les en-tête:

\renewcommand{\sectionmark}[1]{\markright{\small\textit{\thesection~\  #1}}}%Faire apparaître dans les headers les sections en  petit et en italiques
%\renewcommand{\sectionmark}[1]{}%Commenter la lign précédetne et mettre celle-ci pour ne pas avoir le titre des sections dans le header

%% réglages propres à frontmatter

\appto\frontmatter{\pagestyle{fancy}%
	\renewcommand{\chaptermark}[1]{\markboth{\small\textit{#1}}{}}% ne pas faire apparaître de <<numéro>> de chapitre dans les chapitres non numérotés (front: l'introduction, els remerciement, etc)
}

%% réglages propres à mainmatter

\appto\mainmatter{
\renewcommand{\chaptermark}[1]{\markboth{\small\chaptername~\thechapter~--\ \textit{#1}}{}}%faire apparaître dans les headers les sections en  petit et en italiques
%\renewcommand{\chaptermark}[1]{}%Commenter la ligne précédente et mettre celle-ci pour ne pas avoir le titre des chapitres  dans le header
}

%% réglages propres aux annexes

\appto\appendix{
	\renewcommand{\chaptermark}[1]{\markboth{\small~Annexe \thechapter~--\ \textit{#1}}{}}%faire apparaître dans les headers le nom des annexes
	%\renewcommand{\chaptermark}[1]{}%Commenter la ligne précédente et mettre celle-ci pour ne pas avoir le titre des chapitres  dans le header
}




%indiquer des règles d'hyphénation pour des mots précis si besoin
%\begin{hyphenrules}{french}
%	\hyphenation{}
%\end{hyphenrules}


%%%%%%% Package hyperref

% A mettre après les autres appels de packages car redéfinit certaines commandes).

\usepackage[colorlinks=false, breaklinks=true, pdfusetitle, pdfsubject ={Mémoire HN}, pdfkeywords={les mots-clés}]{hyperref} %
\usepackage[numbered]{bookmark}%va avec hyperref; marche mieux pour les signets. l'option numbered: les signets dans le pdf sont numérotés

% Compléter pdfsubjet et pdfkeywords
%Explication des options de hyperref (modifiables)
% hyperindex=false
% colorlinks=false: pour que le cadre des liens n'apparaisse pas à l'impression
% breaklinks permet d'avoir des liens allant sur pusieurs lignes
%pdfusetitle: utiliser \author et \title pour produire le nom et le titre du pdf


%avec overleaf, utiliser :
%\usepackage[xetex]{hyperref}
%\hypersetup{
	%	pdfauthor = {Prénom Nom},
	%	pdftitle = {titre},
	%	pdfsubject = {sujet},
	%	pdfkeywords = {premier mot-clé} {deuxième mot-clé} {troisième mot-clé} {etc}
	%}
\hypersetup{
	breaklinks=true % pour couper le lien sur plusiru lignes
}



%%%%%%%%%%%%%%%%%%%% Package glossaries

%Exception: il faut le charger APRÈS hyperref
%\usepackage[toc=true]{glossaries}
%\makeglossaries
%avec TexStudio: F9 pour compiler le glossaire (s'il y a aussi un index)

%mettre les entrées du glossaire ici ou les mettre dans un fichier à part que l'on appelle ici par \loadglsentries{nom_du_fichier.tex}

%Structure d'une entrée de glossaire
%\newglossaryentry{}{%
%	name={},%
%	description={}
%}



%%%%%%%%%%%%%%%%%% DÉFINITION DES COMMANDES ET ENVIRONNMENTS

\usepackage{tcolorbox}

\newtcolorbox{encadre}[1][]{%
	colback=blue!5!white,
	colframe=blue!75!black,
	fonttitle=\bfseries,
	title=Point technique,
	arc=4pt,
	left=6pt,
	right=6pt,
	top=6pt,
	bottom=6pt,
	fontupper=\fontsize{9}{10}\selectfont
}




 %%%%%%%%%%%%%% INFORMATIONS POUR LA PAGE DE TITRE
 
\author{Joël \textsc{Féral} - M2 TNAH}
\title{L'outil Mezanno pour les approches quantitatives en sciences humaines et sociales. De l'élaboration d'un corpus de documents sériels à l'extraction automatisée de données structurées: les tables annuelles du \hbox{Journal Officiel} (1931 - 1935) comme cas d'usage}

%%%%%%%%%%%%%%%%%%%%%% DOCUMENT
\begin{document}
	\begin{titlepage}
		\begin{center}
			
			\bigskip
			
			\begin{large}				
				ÉCOLE NATIONALE DES CHARTES\\
				UNIVERSITÉ PARIS, SCIENCES \& LETTRES
			\end{large}
			\begin{center}\rule{2cm}{0.02cm}\end{center}
			
			\bigskip
			\bigskip
			\bigskip
			\begin{Large}
				\textbf{Joël Féral}\\
			\end{Large}
		%selon le cas
			\begin{normalsize} \textit{licencié ès lettres}\\
				\textit{diplômé de master}
			\end{normalsize}
			
			\bigskip
			\bigskip
			\bigskip
			
			\begin{Huge}
				\textbf{L'outil Mezanno pour les approches quantitatives en sciences humaines et sociales}\\
			\end{Huge}
			\bigskip
			\bigskip
			\begin{LARGE}
				\textbf{De l'élaboration d'un corpus de documents sériels à l'extraction automatisée de données structurées: les tables annuelles du \hbox{Journal Officiel} (1931 - 1935) comme cas d'usage}\\
			\end{LARGE}
			
			\bigskip
			\bigskip
			\bigskip
			\begin{large}
			\end{large}
			\vfill
			
			\begin{large}
				Mémoire 
				pour le diplôme de master \\
				\enquote{Technologies numériques appliquées à l'histoire} \\
				\bigskip
				2025
			\end{large}
			
		\end{center}
	\end{titlepage}

	\thispagestyle{empty}	
	\cleardoublepage
	
\frontmatter

	\chapter{Résumé}
\medskip
	Résumé du mémoire en français. Cette page ne doit pas dépasser une page.\\
	
	\textbf{Mots-clés:} Journal Officiel; Sénat; Génération structurée; LLM; Mezanno; séparés par des points-virgules.
	
	\textbf{Informations bibliographiques:} Prénom Nom, \textit{Titre du mémoire. Sous-titre du mémoire}, mémoire de master \enquote{Technologies numériques appliquées à l'histoire}, dir. [Noms des directeurs.trices], École nationale des chartes, 20245.
	
		\newpage{\pagestyle{empty}\cleardoublepage}
	
	\chapter{Remerciements}
	
\lettrine{M}es remerciements vont tout d'abord à Jean-Philippe M., Joseph C., Marie P. et Sébastien C. qui m'ont fait confiance pour participer à l'élaboration du projet Mezanno. 

Je remercie également mes parents ET SURTOUT DGETTO !!!
	\newpage{\pagestyle{empty}\cleardoublepage}
	

%%%%%%%%%%%%%%%%%%%%%%%% BIBLIO, BIBLIOGRAPHIE %%%%%%%%%%%%%%%%%%%%%%%%%
%%%%%%%%%%%% \bibliographie ici (normes de l'EnC)

\printbibheading[heading=bibintoc]
\printbibliography[heading=subbibintoc,keyword=primaire,title=Sources primaires]
\printbibliography[heading=subbibintoc,keyword=archivistique,title=Méthodologie historique]
\printbibliography[heading=subbibintoc,keyword=parlementaire,title=Droit et histoire parlementaires]
\printbibliography[heading=subbibintoc,keyword=pat,title=Institutions  et politiques patrimoniales]
\printbibliography[heading=subbibintoc,keyword=hn,title=Humanités numériques]
\printbibliography[heading=subbibintoc,keyword=Numérisation,title=Images et numérisation]
\printbibliography[heading=subbibintoc,keyword=tal,title=Traitement automatique du langage]
\printbibliography[heading=subbibintoc,keyword=archeologie,title=Philosophie de la technique et médialité]
\printbibliography[heading=subbibintoc,keyword=maths,title=Mathématiques]

%%%% réfs oubliées %%%%

%%%
%%%\printbibliography[
%heading=subbibintoc,
%notkeyword=archivistique,
%notkeyword=parlementaire,
%notkeyword=pat,
%notkeyword=hn,
%notkeyword=Numérisation,
%notkeyword=tal,
%notkeyword=archeologie,
%notkeyword=maths,
%notkeyword=primaire,
%title={Oublis}
%]
%%%

%%%%%%%%%%%%%%%%%%%%%%%%%%%%%%%%
%%%%%%%%%%%%%%%%%%%%%%%%%%%%%%%%
%%%%%%%%%%%%%%%%%%%%%%%%%%%%%%%%
%%%%%%%%%%%%%%%%%%%%%%%%%%%%%%%%
%%%%%%%%%%%%%%%%%%%%%%%%%%%%%%%%
%%%%%%%%%%%%BEGIN%%%%%%%%%%%%%%%
%%%%%%%%%%%%%%%%%%%%%%%%%%%%%%%%
%%%%%%%%%%%%%%%%%%%%%%%%%%%%%%%%
%%%%%%%%%%%%%%%%%%%%%%%%%%%%%%%%
%%%%%%%%%%%%%%%%%%%%%%%%%%%%%%%%
%%%%%%%%%%%%%%%%%%%%%%%%%%%%%%%%

\chapter{}
\vspace*{\fill} 
\epigraph{\itshape Remuer du papier ne peut pas être inutile.}{Bruno Latour}

\vfill\clearpage

	
\chapter{Introduction}	
Pour l'historien ou le sociologue, les archives font figure centrale. Pour répondre à une question de recherche, le chercheur va dépouiller, trier, classer, reclasser, recouper les documents ou les données issus de fonds ou de collections recueillies. Cette matière se constitue alors en corpus pour ajuster des questions ou charpenter des réponses. Dans l'épars documentaire s'esquissent des hypothèses, s'élaborent des solutions. L'information enfin extraite des cartons, des rayons ou des documents disponibles sur le Web peut être enfin convoquée, parfois après un laborieux travail de dépouillement. Avec un peu de chance, l'information s'offre presque toute prête : ainsi les lignes budgétaires de tel livre de compte de telle institution; ainsi telles entrées de tels annuaires professionnels. Tantôt il faudra reproduire manuellement l'information qui a été dénichée feuille à feuille et parfois photographiée à la hâte; tantôt, avec un peu plus de chance, simplement récupérer des jeux de données quasiment prêts à l'emploi et les traiter selon ce qu'exige les impératifs scientifiques. Une fois les données réunies en lignes et en tableaux, l'enquête ne fait que commencer: c'est qu'il s'agit de supposer des tendances, des points de comparaison; de dégager aux événements des séries ou des structures explicatives et forger des réponses. Ainsi la "centralité de l'archive" -- et des données -- dans le travail de l'historien -- hier Lucien Febvre, Prost.. -- ou du sociologue -- Durkheim -- ou encore, pourquoi pas, de l'astronome et ses archives voulant restituer un passé aux trajectoires des comètes. 

Derrière ces masses de documents sériels, une allure : ce sont des bases de données de papier. Elles contiennent des noms et des nombres, des métiers ou des adresses. Pour le chercheur, ces sources répétitives se prêtent volontiers à une traduction numérique pour faciliter et opérer des traitements plus systématiques en bénéficiant des capacités calculatoires de l'ordinateur. Egalement, pour l'archiviste en charge de l'indexation des fonds, elles regorgent de noms qu'il est intéressant d'extraire et de fournir, par exemple, à son public de généalogistes. Naturellement, il est tentent d'employer l'outil informatique pour déléguer -- ou rendre possible -- ce travail d'extraction et de structuration de l'information présente dans les masses documentaires. Les grands modèles de langue (les \textit{LLM}) semble pouvoir faire endosser aux ordinateurs ce labeur de traduction et de structuration de l'information et sans dépendre en amont de données déjà bien formées.

Ce mémoire interroge la problématique de la traduction de ces corpus sériels en ensembles de données structurées, dans une perspective d’analyse historienne et archivistique. 
Historienne d'abord, car il s'agit d'explorer de "nouvelles frontières" disciplinaires impliquant la collaboration entre chercheurs en SHS, informaticiens et institutions patrimoniales en vue d'exploiter à des fins scientifiques des données issus de fonds numérisés; archivistique ensuite, car la traduction des informations contenues dans les documents en données exploitables par des systèmes informatiques rejoignent les enjeux d'indexation et donc de valorisation d'ensembles documentaires.

Cette problématique de traduction de fonds sériels en données structurées exploitables pour l'analyse quantitative historienne ou pour la valorisation documentaire part d’un constat : nombre de publications administratives ou normatives — annuaires, lois, décrets, tables parlementaires — relèvent d’une production sérielle à forte teneur informationnelle mais échappent aux catégories sensibles habituellement mobilisées dans le rapport aux archives. Leur lecture manuelle est difficile, leur dépouillement peut être décourageant. On s'éloigne ici du "goût de l'archive" d'Arlette Farge qui dépeint une phénoménologie sensible de la source historique pour adopter une approche moins solipsiste de valorisation de documents "sans goût" -- mais dont on aura restitué un pluriel. Ces "fonds dormants" -- ou du moins ces documents ingrats qu'il est difficile d'exposer étant donné leur austérité -- forment en effet une mémoire institutionnelle précieuse qu'il convient d'interroger dès lors qu’on parvient à les structurer et les croiser avec d’autres sources.

Si on n'arrête pas de louer depuis quelques décennies de nouveaux tournants dans la façon de travailler sur les sources grâce à l'outil numérique, il faut bien avouer que les derniers développements autour des grands modèles de langage accentuent un virage. La fouille de texte, la \enquote{lecture à distance} (\textit{distance reading}), et tout ce qui implique l'extraction d'information sémantique, reposent sur une récente synergisation des techniques numériques : tout d'abord, la capacité à restituer un objet physique en image discrète (numérisation); de structurer et normaliser l'accès à des images dans des dépôts centralisés via des protocoles HTTP(API); de retranscrire de l'image l'information textuelle (OCR) ou des structures de pages (Document Layout Detection); et de la structuration sémantique \textit{a posteriori} des entités nommées de façon semi-automatique -- moyennement un apprentissage supervisé via l'annotation ou le moissonnage de la production scientifique en ligne, déjà numérisée. Cette synergie des traitements sur l'image numérique et des développement en linguistique computationnelle renouvelle \enquote{les frontières de l'historien} en ce sens que, pour peu que les documents soient numérisées, on puisse automatiser le travail d'extraction de l'information présente dans les sources. La chaîne qui va de la numérisation du document à l'extraction de données sémantiquement structurées et son analyse via l'outil informatique forme un \textit{système technique} (Simondon) et l'historien, le sociologue ou encore le statisticien s'y inscrivent pleinement. 

Une précaution : la question technique qui participe de l'administration des nouvelles frontières de l'historien n'est pas seulement un moyen d'automatiser des tâches qui autrefois étaient plus manuelles : elle apporte dans ses rêts de nouveaux enjeux épistémologiques (les données ne se donnent pas, ce sont des \textit{capta}, Drucker), des manières de réactiver des questions ou des réponses (Colingwood), des sources documentaires, des méthodes (impliquant un redivision du travail) ou des façons de travailler (Ladurie, l'historien et l'ordinateur). Les questions techniques sont moins une affaire d'automatisation que de sensibilité des systèmes techniques à l'information et leur relation avec le travail humain (Simondon). De même, les instruments ne font pas que servir le travail intellectuel de façon neutre : ils sont aussi de la \enquote{théorie réifiée} (Bachelard). Se superposent ainsi le champ énonciatif des systèmes techniques et des disciplines scientifiques qui s'y inscrivent et expriment des connaissances \textit{situées}. Les outils techniques promettent des méthodologies -- lesquelles reposent sur de manière d'envisager la division d'un travail pour une tâche déterminée. Les \textit{pipelines} de traitement de données -- de l'image au texte structuré -- se constituent à la fois comme outil et comme condensation \textit{in silicium} des \textit{a priori} épistémologiques. Les solutions techniques à des problèmes scientifiques sont une affaire de \textit{design} (Anne-Lyse Renon), c'est-à-dire de stratégies épistémiques et valuatives.


Dès lors, comment articuler les enjeux historiographiques propres à ces corpus avec les nouvelles opportunités d’extraction et de structuration automatiques qu’offrent les techniques récentes et leur synergisation ? (OCR, modèles de langage, outils d’annotation) ? Comment construire une chaîne de traitement reproductible, capable de restituer la richesse de ces fonds tout en garantissant la qualité des données produites ? La question de la valuation des méthodes -- c'est-à-dire de leur légitimité au regard de ce à quoi on tient savoir --, elle-même, semble dépendre des besoins : un archiviste n'a pas tout à fait les mêmes besoins qu'un historien.



Ces questionnements auront été les miens durant mon stage à l'EPITA/BnF où j'ai eu l'occasion de travailler à la convergence des nouvelles opportunités de traitements automatiques et de la question de la légitimité épistémique de données produites par des systèmes techniques, à partir de l'extraction d'informations sous forme structurée du Journal Officiel de 1931, dans une optique de préparation de l'analyse des discours historiques sous la IIIe République. [à développer] 

Cette interrogation se décline ainsi selon ces trois axes :

\begin{itemize}
	\item Que sont les sources sérielles et quels sont leurs apports spécifiques pour répondre à une question de recherche ? En quoi leur structure, leur cumulativité et leur forme normative permettent-elles des lectures nouvelles, notamment en lien avec les pratiques de gouvernement et les dispositifs de publicité du droit sous la IIIe République ? Pour aborder ce point, je me pencherai sur l'analyse des Tables du Sénat de 1931 car, dans une perspective de \textit{reenactement} historien de l'activité parlementaire, elles donneraient un corps à cette problématique de traduction de sources sérielles en sources exploitables pour l'analyse.
	\item Quelles méthodes et quels outils permettent aujourd’hui d’en automatiser la structuration ? Comment construire un protocole d’extraction cohérent, tenant compte de la matérialité des documents (OCR, mise en page, bruit), des formats de sortie, et des finalités analytiques visées ?
	\item Comment évaluer la fiabilité des données ainsi produites et garantir leur légitimité scientifique ? Quels critères de qualité, de traçabilité et de transparence permettent de faire de ces résultats des objets d’enquête mobilisables par les historiens ?
\end{itemize}

A travers ces trois axes, qui constituent chacun une partie de ce mémoire, je veux donc répondre à la problématique du passage de l'information sérielle non-structurée présente dans les sources, à la fois sur des aspects documentaires; techniques (méthodes d'extraction) et valuatives (évaluation et épistémologie de l'évaluation).



\newpage{\pagestyle{empty}\cleardoublepage}

				%%%%%%%%%%%%%%%%%	Le corps du mémoire

\mainmatter

%%%%%%%%%%%% PARTIE 1
\part{Des sources sérielles: les Tables Annuelles du Sénat comme cas d'usage}

\chapter{Le \emph{Journal Officiel} : la parole, la main, la vue et le droit}

\enquote{Pas de question sans document. [La question de l'historien] n'est pas une question nue; c'est une question armée, qui porte avec elle une idée des sources documentaires et des procédures de recherches possibles}. \footcite[][80]{prost}. Ainsi, Antoine Prost, à partir des propositions de Robin G. Collingwood sur les questions historiques, expose l'idée d'une solidarité entre une question de recherche, les documents et les procédures de leur traitement. La question de l'historien, tout d'abord, est \enquote{armée} : cela signifie qu'elle ne vient pas seule pour satisfaire, avec désintéressement, une curiosité passagère; elle prend appui sur l'actualité de l'historien, de ce qui fait qu'elle puisse être pertinente socialement et scientifiquement. \enquote{Armée} : cela veut dire également que l'historien a déjà une idée des sources documentaires sur lesquelles travailler. Le chemin de la connaissance du passé, qui s'ouvre avec une question de recherche, est motivé par un \emph{reenactement}, une réactivation du passé depuis le présent. Par là, il s'agit d'être à l'écoute des \enquote{palpitations du temps} présent -- pour reprendre ce mot de l'historien de l'art Eugeni d'Ors \footcite[][]{ors} -- et de réinterroger l'actualité à travers des documents à explorer. \enquote{Il n'y a d'histoire que de choses pensées au présent par l'historien}\footcite[][p.166]{prost} : selon l'approche collingwoodienne -- qui n'est d'ailleurs pas sans rappeler la position de Walter Benjamin sur l'histoire comme \enquote{objet d’une construction dont le lieu n’est pas le temps homogène et vide, mais qui forme celui qui est plein de temps actuel} \footcite[][]{benjamin} -- la pratique historienne est une dialectique entre passé et présent. Les documents, alors, ne sont pas \enquote{un dépôt mort} mais une \enquote{énergie fossile} qui reffectue une pensée\footcite[][]{libera} et viennent renouveler un questionnaire. Les prétentions historiennes collingwoodiennes semblent alors aller au-delà d'une satisfaction strictement érudite -- ou du moins sans volitions éthiques -- pour atteindre la nécessité de réévaluer le passé avec des considérations toutes pragmatistes, c'est-à-dire valuatives\footcite[][]{dewey}. Si l'historien est lui-même un sujet situé historiquement et socialement, il est également saisi dans des configurations techniques qui peuvent changer ses procédures de recherche -- la numérisation de sources historiques et leur mise à disposition sur le Web étant un exemple assez criant. Le milieu de l'historien a donc lui-même une historicité; sociale, technique, psychique --  car \enquote{il n'y a pas d'histoire sans préjugés} selon les mots de Francis H. Bradley, rapportés par Antoine Prost\footcite[][p. 96]{prost}.

À l’heure où la question de la dette occupe le centre des débats publics, il est pertinent de réactiver une pensée historique pour éclairer les inquiétudes du présent. L’analyse des débats parlementaires d’autres périodes de crise, saisis à travers le \emph{Journal officiel} et des instruments de repérage comme les tables nominatives, offre une occasion de réfléchir à la qualité des échanges démocratiques et aux formes de polarisation politique qui traversent nos institutions. Ainsi, les controverses actuelles sur \enquote{l’ensauvagement} du débat parlementaire\footcite[][]{ensauvagement}, ou encore sur la \enquote{polarisation}\footcite[][]{polarisation} politique française dans un cadre international sous tension, ne se comprennent pleinement que replacées dans une histoire longue des pratiques délibératives et des dynamiques institutionnelles. Le \emph{Journal officiel}, qui restitue l'activité parlementaire, accompagne donc la question de la polarisation politique. Ce travail engage à une double exigence : d’une part, saisir la singularité des crises d’hier -- comme celles économiques de 1931 -- dans leurs propres rationalités politiques et sociales ; d’autre part, réfléchir aux continuités et ruptures qui relient ces débats à notre époque, à travers la médiation des sources et des techniques de recherche actuelles. Le \emph{reenactment} collingwoodien prend ici toute sa portée : relire le passé n’est pas reproduire mécaniquement des faits, mais réactualiser une pensée inscrite dans les documents avec les nouvelles opportunités techniques d'exploration des sources pour nourrir un questionnement critique sur le présent.

Dans cette partie, cette question de la polarisation se fait point de départ pour une investigation sur les procédures techniques d’exploration des sources sérielles. Si la question historienne est aussi un dialogue avec des sources, elle est également médiatisée par les méthodes employées pour les interroger et par les relations qu’elles peuvent tisser entre elles. Il faut donc comprendre \emph{comment} la source est fabriquée, ce qu’elle est capable de restituer ou, au contraire, de rendre absent ; analyser également sa place dans un « énoncé »\footcite[][]{Foucault}, en interrogeant sa position disciplinaire -- les numéros du \emph{Journal Officiel} sont des archives ou de la documentation ? Qu'est-ce que cette classification peut dire de son rôle ? -- ; et enfin étudier son fonctionnement prosaïque, c’est-à-dire saisir l’écologie documentaire dans laquelle est s'inscrit. Nous commencerons par dessiner une histoire synthétique de la transcription des débats parlementaire et de l'édition des lois publiées -- du \emph{Moniteur universel} au \emph{Journal Officiel} en passant par \emph{Le Bulletin des lois} -- puis nous donnerons plus d'attention au \emph{Journal Officiel} dans les années 30, à partir des archives de la \emph{Direction du Journal Officiel}, administration du Ministère de l'Intérieur.

\section{Le \emph{Journal Officiel} : le droit et la main}

Le \emph{Journal Officiel} est l’aboutissement d’une longue évolution des supports de publicisation des débats parlementaires et des textes législatifs en France. S'il né en 1869, sous le Second Empire, c'est en remplacement du \emph{Moniteur universel}, lui-même héritier des ambitions révolutionnaires. Son histoire reflète cependant une transformation plus large : celui du passage d’une logique où la transparence des débats publics à l'Assemblée nationale était un idéal politique à une ère où la publication des textes législatifs devient un acte juridique fondateur. Cette métamorphose, qui s’étend sur plus de deux siècles, illustre le passage d’une logique de transparence, forgé par un idéal politique révolutionnaire; à une ère où la publication devient un acte performatif, indissociable de la validité des textes législatifs.

\subsection{La Révolution française : l’émergence d’une publicité politique (1789–1799)}

La Révolution française marque une rupture radicale en érigeant la publicité des débats en principe constitutionnel. La Constitution de 1791 proclame ainsi que \emph{« les délibérations du Corps législatif seront publiques, et les procès-verbaux de ses séances seront imprimés »}\footcite[][]{constitution} actant la nécessité de rendre compte au peuple des discussions qui engagent son destin. Cette exigence de transparence se concrétise par la création de deux organes distincts, répondant à des logiques complémentaires mais distinctes : le \emph{Moniteur universel}, initiative privée -- et le \emph{Bulletin des lois} -- initiative du Directoire qui, selon Frédéric Graber, a servi à \enquote{forcer le consentement} des citoyens à la loi \footcite[][]{graber}. 

Le \emph{Moniteur universel}, fondé en 1789 par Charles-Joseph Panckoucke, s’impose rapidement comme le premier vecteur de publicisation des débats parlementaires et qui connut un \enquote{vif succès public} car produisant \enquote{un fort effet de réel} \footcite[][]{coniez}. Son ambition affichée est de retranscrire les discussions de l’Assemblée nationale, mais ce travail de transcription, loin d’être neutre, relève déjà d’une reconstruction orientée. Comme le souligne Benjamin Morel, ces comptes rendus n’étaient pas de simples \emph{verbatims} : ils reflétaient les lignes de force politiques de l’époque, tout en servant un objectif plus large, celui de rendre publique une parole parlementaire en construction. Par exemple, les journalistes du \emph{Moniteur}, comme Hugues-Bernard Maret, devaient résumer ou interpréter les propos en fonction de leurs affinités politiques, mais aussi des contraintes techniques de l’époque\footcite[][]{coniez}. Les débuts de la transcription étaient en effet artisanaux : les rédacteurs prenaient des notes à la main, ce qui rendait les comptes rendus souvent incomplets ou biaisés. Dès l’origine, il s’agissait de produire un texte lisible, adapté à un public élargi, tout en servant les intérêts d’une institution en quête de légitimité\footcite[][]{morel}. Une innovation notable fut le \emph{Journal logographique} (1791), créé par Le Hodey de Saultchevreuil, qui utilisait une méthode de rotation entre plusieurs rédacteurs pour capturer l’intégralité des débats, utilisant un système de bande de papier étroites et allongées. Chaque \enquote{logographe} prenait des notes pendant quelques minutes avant de passer le relais, permettant une couverture plus complète. Cependant, cette méthode, bien qu’ingénieuse, échouait à restituer la plurivocité des échanges, réduisant la complexité des débats à une succession de discours linéaires\footcite[][]{coniez}.

Parallèlement, le \emph{Bulletin des lois}, créé en 1793, remplit une fonction radicalement différente. Contrairement au \emph{Moniteur}, qui se concentre sur les débats, le \emph{Bulletin} a pour vocation exclusive de publier les lois et décrets adoptés par les assemblées révolutionnaires\footcite[][]{saudrais}. Son objectif n’est pas d’informer ou d’éduquer, mais de rendre opposables les textes législatifs en les portant à la connaissance des citoyens et des autorités locales. La publication est un relais d’information législative puis, sous le Directoire, un acte fondateur de la validité juridique : une loi non publiée n’existe pas en droit. Cette distinction fondamentale entre les deux organes – l’un tourné vers la délibération, l’autre vers la norme – préfigure la dualité fonctionnelle qui caractérisera plus tard le \emph{Journal Officiel}. 

Ainsi, dès ses origines, la publicisation des débats et des textes législatifs répond à deux logiques distinctes, voire antagonistes : une logique démocratique, où la transparence vise à éclairer le citoyen et à légitimer le pouvoir par le débat ; et une logique juridique, où la publication conditionne l’effectivité des normes. Ces deux dimensions, bien que coexistantes, restent cependant séparées jusqu’au milieu du XIXe siècle.

\subsection{Le XIXe siècle : vers une institutionnalisation de la transparence (1800–1870)}

Au cours du XIXe siècle, les régimes successifs – Consulat, Empire, Restauration, Monarchie de Juillet, Seconde République – maintiennent une publicité contrôlée des débats parlementaires, mais deux évolutions majeures se dessinent, qui vont progressivement rapprocher les fonctions de publicisation et de légalisation. D’une part, la professionnalisation des comptes rendus, avec la professionalisation de la sténographie et ses méthodes de retranscription plus fidèles et systématiques des débats\footcite[][]{gardey}. D’autre part, la fusion progressive des organes de publication prépare le terrain à l’émergence d’un support unique, combinant les héritages du \emph{Moniteur} et du \emph{Bulletin des lois}.

Sous le Consulat et l’Empire, le \emph{Moniteur universel} perd son indépendance pour devenir un organe de propagande au service du pouvoir napoléonien\footcite[][]{universalis}. Les comptes rendus y sont épurés, voire censurés, et la transparence n’y est plus un idéal démocratique, mais un instrument de légitimation du régime. Cette période marque un recul temporaire dans la publicisation des débats, mais elle ne remet pas en cause la nécessité structurelle de rendre compte, ne serait-ce que de manière contrôlée, des travaux parlementaires. 

La Restauration et la Monarchie de Juillet voient cependant un renouveau de la publicité parlementaire, porté par deux dynamiques complémentaires. D’une part, les progrès techniques, sous l'impulsion des services de sténographie des chambres d'un Prévots et d'un Célestin Lagache, permettent une retranscription plus précise des débats\footcite[][]{coniez}. En 1831, Jean-Baptiste Breton fonde le \emph{Sténographe des Chambres}, première tentative de compte rendu intégral fondé sur une méthode systématique de prise de notes\footcite[][]{coniez}. Bien que cette initiative échoue faute de moyens -- car sa méthode sténographique impliquait un dispositif couteux, basé sur la relève régulière des sténographes, chacun étant une tête à rémunérer --, elle préfigure la professionnalisation des rédacteurs parlementaires. D’autre part, la montée en puissance des assemblées sous la Monarchie de Juillet conduit à une demande croissante de transparence. En 1848, l’Assemblée nationale intègre les sténographes dans ses services, marquant la naissance d’un corps de fonctionnaires dédiés à la retranscription des débats. Le compte rendu cesse d’être une initiative privée pour devenir un document officiel, garanti par l’État. la révolution dans la transcription des débats parlementaires est venue avec le développement de la sténographie. Si cette technique, introduite en France à la fin du XVIIIe siècle, consistait à utiliser des signes abrégés pour noter la parole à la vitesse à laquelle elle était prononcée, la sténographie n’était pas seulement un outil technique, mais un instrument politique : elle permettait en effet de transformer une parole orale, souvent chaotique, en un discours écrit structuré, parfois conforme aux attentes d’un État en quête d’ordre et de rationalité\footcite[][]{morel}. Avec l'intègration officielle des sténographes dans les services parlementaires, vient à naître un corps de sténographes d’État. Cette institutionnalisation répondait à un double impératif : assurer la publicité des débats -- un principe constitutionnel -- et contrôler la production du compte rendu pour en faire un outil au service de la légitimité parlementaire \emph{(Gardey, 2010)}.  Cette institutionnalisation s’accompagne donc d’une codification stricte des méthodes de rédaction. Les sténographes, désormais soumis à des règles précises, ne se contentent plus de transcrire, mais recomposent les discours pour les adapter aux exigences de la langue écrite et aux attentes d’un lectorat élargi (Prévost, 1848). Comme l’a souligné Hippolyte Prévost, leur travail relève moins de la transcription littérale que de la traduction : il s’agit de « rendre justice à l’orateur » tout en produisant un texte « clair, cohérent et conforme aux canons de l’écrit » \footcite[][]{morel}. Cette approche, qui privilégie la lisibilité à la fidélité absolue, sera reprise et systématisée par les services officiels du \emph{Journal Officiel}.Parallèlement, le \emph{Bulletin des lois} poursuit sa mission de publication des textes normatifs, mais son rôle se trouve progressivement absorbé par celui du \emph{Moniteur}, qui, sous la Seconde République, commence à intégrer des extraits législatifs. Cette convergence prépare le terrain à la fusion des deux fonctions – information et légalisation – qui interviendra à la toute fin du Second Empire.

La création du \emph{Journal officiel de l’Empire français} en 1869, sous Napoléon III, marque un tournant décisif. Pour la première fois, un seul organe rassemble les débats parlementaires, héritiers du \emph{Moniteur universel}, et les lois et décrets, héritiers du \emph{Bulletin des lois}. L’adjudication du 24 septembre 1868, publiée dans le \emph{Journal officiel} du lendemain, attribue « le droit exclusif d’imprimer et de publier les deux journaux officiels paraissant sous le titre de \emph{Moniteur universel} » à l’imprimeur Wittersheim au 31 quai Voltaire à Paris, à proximité du Sénat et du Corps législatif -- la publication comprenant également l'envoi à la poste, la désignation des \enquote{écrivains} -- des rédacteurs du \emph{Journal Officiel} -- étant une prérogative gouvernementale\footcite[][]{cote1}. L'adjudicataire sera ainsi tenu de \enquote{publier chaque jour [...] les comptes rendus sténographiques des séances du Sénat et du Coprs législatif, les lois, décrets et actes officiels, judiciaires ou administrative dont l'insertion est réclamée par le Gouvernement}\footcite[][]{cote1}. Ce texte place celui-ci sous la tutelle du ministre Ernest Pinard, soumet la désignation des rédacteurs à l’approbation de l’exécutif et prévoit que, en cas de manquement, le contrat peut être résilié par voie de régie\footcite[][]{cote1}, sous les douze ans. Le premier numéro sera publié le 1er janvier 1869 sous le nom de \emph{Journal Officiel de l'Empire français} et sera renommé en 1870 \emph{Journal Officiel de la République française} à la chute du régime impérial. En 1870 commence réellement une transition administrative et organisationnelle, suite à l'instabilité politique et la guerre -- l'envoi du \emph{Journal Officiel}, pendant le siège de Paris de 1871 sera d'ailleurs envoyé expédié par ballon; les bureaux et imprimerie déménagés à Tours sous la Délégation du Gouvernement de la Défense Nationale puis à Versailles jusqu'en 1879\footcite[][]{cote1}. En 1880, avec la loi du 28 décembre, est créee l'administration du \emph{Journal Officiel}, placée sous la tuelle du Ministère de l'Intérieur. C'est une société anonyme de composition -- la SACIJO, (Société Anonyme de Composition et d'Impression du \emph{Journal Officiel}) -- qui sera en charge de l'impression.

Cette fusion des comptes rendus et publication des actes officiels n’est pas anodine : elle signe en fait la réconciliation des deux logiques qui, depuis 1789, coexistaient sans se confondre. Désormais, la publicisation des débats et la publication des textes normatifs ne sont plus deux missions distinctes, mais les deux faces d’une même entreprise : éclairer le citoyen tout en fondant la légalité des actes. Avec la chute du Second Empire et l’avènement de la IIIe République en 1870 --- le journal devenant le \emph{Journal officiel de la République française} -- consacre une nouvelle ère dans l’histoire de la publicité parlementaire. 

\subsection{Le \emph{Journal Officiel} dans les années 30}

Au début du \emph{Journal Officiel}, en 

\section{Une histoire du \emph{Journal Officiel} par sa nature documentaire : archives ou documentation ?}

\begin{quote}
Ne pas faire une histoire du \emph{Journal Officiel}, de ses origines; mais une histoire dans sa position dans un énoncé à travers les façons de le classer.

\end{quote}
Plutôt que de retracer de manière exhaustive l’histoire éditoriale du \emph{Journal Officiel} depuis sa création en 1869, il s’agit ici d’interroger sa place dans les dispositifs documentaires et archivistiques. Cette approche consiste à lire le J.O. non comme un simple périodique institutionnel, mais comme un objet documentaire dont le classement, les circuits de diffusion et les conditions de conservation reflètent des enjeux politiques et symboliques.

Dès ses débuts, le \emph{Journal Officiel} est un produit éditorial : rédigé par des sténographes et des correcteurs, mis en forme par des linotypistes et des imprimeurs, vendu par abonnement, il relève autant du monde de la presse que de celui de l’administration. Soumis au dépôt légal, il est conservé à la Bibliothèque nationale de France, aux bibliothèques de l’Assemblée nationale et du Sénat, et dans les bibliothèques des Archives nationales. Cette logique de diffusion correspond à sa vocation première : rendre publique la loi et l’activité parlementaire, garantir leur accessibilité.

Pourtant, le J.O. occupe également une place inattendue dans les Archives départementales, au sein de la série K (« Lois, ordonnances et arrêtés »), créée en 1841. Cette série, conçue pour rassembler les actes de l’État et les publications officielles postérieures à la Révolution, accueille non seulement les arrêtés préfectoraux, mais aussi des périodiques comme \emph{Le Moniteur universel}, le \emph{Bulletin des lois}, et le \emph{Journal Officiel}. Or, ce classement brouille la distinction classique entre archives (documents produits par une activité administrative et conservés comme traces) et documentation (objets édités, destinés à la lecture publique).

L’inclusion du J.O. dans la série K révèle ainsi une stratégie politique de diffusion : garantir que chaque département dispose d’un exemplaire des textes législatifs et réglementaires, afin de renforcer l’autorité centrale sur l’ensemble du territoire. Plus qu’un simple effet de classement, ce choix traduit une conception du document comme instrument de gouvernement. Comme l’écrivait Natalis de Wailly en 1841, la série K devait être le complément moderne de la série A (édits et ordonnances royales de l’Ancien Régime), actualisant la même logique de centralisation et de publicité des décisions de l’État.

Cette situation hybride du J.O. — à la fois publication soumise au dépôt légal et archive départementale — souligne la complexité des catégories documentaires. Des archives ? Le J.O. ne correspond pas à la définition juridique stricte des archives (Code du patrimoine, art. L.211-1) puisqu’il ne constitue pas un produit « organique » d’une activité administrative, mais un travail éditorial destiné au public. De la documentation ? Il est pourtant intégré au cadre de classement des archives publiques, et traité comme un témoin administratif de l’action politique, au même titre qu’un arrêté préfectoral.

Ce double statut montre que la frontière entre archive et publication est poreuse. Le J.O. matérialise le lien entre la production politique et sa diffusion normative : il est à la fois trace administrative et objet éditorial. Sa conservation dans la série K incarne une volonté d’accessibilité universelle de la loi, qui dépasse la seule logique de préservation patrimoniale.

En ce sens, l’histoire du J.O. se comprend moins comme une succession d’évolutions éditoriales que comme une inscription dans un énoncé documentaire : une architecture de classement, de conservation et de circulation qui place ce périodique au cœur de la mémoire politique française. Cette perspective est indispensable pour saisir la nature des \emph{tables annuelles}, elles-mêmes produites dans une logique de sérialisation et de diffusion, et qui traduisent en indexation ce même souci de rendre le droit et la parole parlementaire accessibles et exploitables.

\section{Le \emph{Journal Officiel} : entre publicité et promulgation}

La création du \emph{Journal Officiel de la République française} en 1869 s’inscrit dans une longue histoire des dispositifs de publicité de la loi. Héritier à la fois du \emph{Moniteur universel} (1789–1869), qui transcrivait les débats parlementaires, et du \emph{Bulletin des lois} (1793–1931), garant de la promulgation exécutoire des textes normatifs, le \emph{Journal Officiel} cumule deux fonctions : informer le public des débats parlementaires et conférer force obligatoire aux lois par leur publication. Cette double vocation, attestée dès la Révolution française, fait de ce périodique un instrument essentiel de la transparence parlementaire et de l’effectivité juridique.

Son caractère sériel — une parution quasi quotidienne, au format stable, avec une organisation régulière des rubriques — constitue précisément ce qui en fait une source de premier plan pour l’historien. Sa régularité et sa cumulativité permettent de suivre à la trace l’activité parlementaire, mais elles posent aussi la question de la matérialité du texte publié : transcription fidèle ou recomposition éditoriale ? Dès lors, le \emph{Journal Officiel} ne doit pas être lu seulement comme un réceptacle de la parole parlementaire, mais comme une œuvre éditoriale, façonnée par les sténographes, rédacteurs et correcteurs qui traduisent l’oralité en texte écrit.

La nature archivistique du \emph{Journal Officiel} demeure ambivalente. Juridiquement, il s’agit d’une publication soumise au dépôt légal, conservée à ce titre à la Bibliothèque nationale de France, à la bibliothèque des Assemblées et aux Archives nationales. Pourtant, son intégration dans la \textbf{série K des Archives départementales} interroge : peut-on considérer comme archives des documents produits à des fins éditoriales, vendus au public et destinés à être lus ?

L’analyse archivistique rappelle que les archives sont le « collatéral direct » d’une activité administrative, produites sans intention de publication. Or, le \emph{Journal Officiel} est un objet éditorial dont la finalité première est précisément la publicité. Son inscription en série K — aux côtés des lois, ordonnances et arrêtés préfectoraux — illustre néanmoins la manière dont l’État a voulu garantir, dans chaque département, l’accès des citoyens aux textes normatifs. Loin d’être un simple effet du « respect des fonds », cette intégration témoigne d’une stratégie politique de diffusion centralisée du droit.

Cette tension entre document édité et document d’archives souligne la spécificité des sources sérielles : elles se situent à la frontière entre mémoire administrative et communication publique, entre traces institutionnelles et dispositifs de légitimation.

\section{Le \emph{Journal Officiel} : un contexte technique et administratif (1921–1940)}

Pour comprendre la matérialité de la source exploitée dans ce mémoire (les tables du Sénat de 1931), il faut rappeler que le \emph{Journal Officiel} n’était pas seulement le produit d’une Chambre parlementaire, mais celui d’une organisation industrielle. Depuis 1880, la publication est assurée par la \textbf{Société anonyme coopérative de composition et d’impression des Journaux officiels (SACIJO)}, placée sous tutelle du ministère de l’Intérieur. Linotypistes, rotativistes, correcteurs et personnels administratifs concourent à sa fabrication quotidienne.

La période 1921–1940, bornée par l’acquisition de la \emph{linotype Model 9} et l’interruption de 1940, offre un cadre technique relativement homogène. Elle correspond aussi à une stabilité des pratiques éditoriales, qui permet d’envisager une lecture sérielle. Les archives de la Direction des Journaux officiels (série 19840069 des AN) éclairent cette fabrique administrative : rapports budgétaires, organigrammes, correspondances de service. Elles révèlent un fonctionnement hybride, entre administration d’État et entreprise de presse, qui explique la diffusion massive et régulière de ces volumes.

\section{Les « processus métier » de la publicité parlementaire à partir des Tables nominales : analyse des sources}

\begin{quote}
Schémas en simili-rdf 

\end{quote}
Qualifier la chaîne de production parlementaire en termes de « processus métier » revient à cartographier l’ensemble des opérations qui transforment la parole politique en texte normatif publié. À la IIIe République, ce processus suit plusieurs étapes :

\begin{itemize}
\item \textbf{Délibérer} : débats oraux au Sénat et à la Chambre des députés, régis par les règlements de 1876, avec une organisation en bureaux et commissions.
\end{itemize}
\emph{ \textbf{Transcrire} : sténographes et rédacteurs produisent les comptes rendus }in extenso*, qui passent par un travail de révision avant impression.
\emph{ \textbf{Publier} : les textes sont édités dans le }Journal Officiel* et diffusés par abonnement et dépôt légal.
\begin{itemize}
\item \textbf{Promulguer} : la publication confère force obligatoire aux lois votées, qui ne prennent effet qu’une fois rendues publiques.

\end{itemize}
Ce continuum — délibérer, transcrire, publier, promulguer — rend manifeste le rôle central du \emph{Journal Officiel}. Il ne s’agit pas seulement d’un témoin documentaire, mais d’un maillon de l’effectivité du droit.

\chapter{Les tables annuelles : des relations documentaires}

\section{Les tables dans l’environnement du \emph{Journal Officiel}}

À côté des livraisons quotidiennes du \emph{Journal Officiel}, le dispositif documentaire de la Troisième République produit un ensemble d’outils de repérage et de cumul : index, tables et recueils annuels. Ces tables, organisées par Chambre et par type de document (séances, questions, interventions, lois, décrets, etc.), constituent un instrument de navigation à travers la masse documentaire accumulée. Elles offrent un second niveau de structuration, indispensable à l’exploitation d’un corpus qui, sans cela, serait pratiquement illisible dans son entier.

Dans ce sens, les tables ne sont pas de simples annexes, mais un élément constitutif du \emph{Journal Officiel}. Leur publication témoigne d’une volonté de rendre praticable la lecture sérielle, en transformant un flot continu de débats en une matière consultable a posteriori. Elles permettent aux parlementaires, aux fonctionnaires et aux juristes, mais aussi aux journalistes et au public, de retrouver un débat, une loi ou un orateur dans un ensemble potentiellement infini de pages.

\section{Forme et organisation des tables}

La table annuelle se présente comme un volume imprimé, distinct des numéros quotidiens mais reprenant la même logique typographique de sobriété. La structuration est généralement alphabétique ou thématique, avec des entrées renvoyant à des numéros de séance ou de page du \emph{Journal Officiel}. Ainsi, le chercheur y trouve à la fois :

\begin{itemize}
\item des index de noms (parlementaires, ministres, orateurs) ;
\item des index de matières (projets de lois, sujets débattus, thèmes abordés) ;
\item des références législatives (dates, intitulés, numéros de lois et décrets).

\end{itemize}
Cette composition apparemment simple reflète un travail complexe de collecte et de mise en ordre, qui engage des méthodes d’indexation encore largement manuelles dans les années 1930. Les tables matérialisent donc une double médiation : celle de la transcription sténographique, puis celle de la mise en indexation.

\section{Informations sémantiques et usages}

Les tables ne livrent pas seulement des renvois. Leur organisation alphabétique ou thématique suggère déjà une lecture orientée du corpus. En réordonnant les débats selon les sujets ou les personnes, elles produisent une représentation « secondaire » de l’activité parlementaire :

\begin{itemize}
\item \textbf{Pour l’historien}, elles permettent de cartographier les thèmes récurrents, d’identifier des trajectoires individuelles de parlementaires, ou encore de suivre la maturation d’une question dans le temps long.
\item \textbf{Pour les juristes}, elles assurent un repérage efficace des textes normatifs, condition de la sécurité juridique.
\item \textbf{Pour l’administration}, elles facilitent la réutilisation interne des débats et la circulation de l’information entre services.

\end{itemize}
En ce sens, les tables possèdent une valeur sémantique propre : elles ne sont pas de simples index, mais des instruments de catégorisation, qui hiérarchisent les contenus du \emph{Journal Officiel} et leur confèrent une visibilité inégale.

Les tables comme « hub » intercorpus

Enfin, les tables établissent des liens entre différents ensembles documentaires. Elles ne se limitent pas aux seuls débats parlementaires, mais relient ceux-ci aux autres publications officielles et à des corpus complémentaires. Elles servent d’articulation entre :

\emph{ les volumes quotidiens du }Journal Officiel* ;
\emph{ les recueils législatifs et réglementaires (par exemple le }Bulletin des lois*) ;
\begin{itemize}
\item les instruments internes des Chambres (procès-verbaux, rapports de commissions) ;
\item les archives départementales (série K), où elles prennent place aux côtés d’autres formes de publicité administrative.

\end{itemize}
En occupant cette position nodale, les tables fonctionnent comme des « hubs documentaires » : elles permettent de passer d’un corpus à l’autre, et d’inscrire les débats dans l’écosystème plus large des pratiques de gouvernement.

Exemple : Les Tables du Sénat, année 1931

Le volume des \emph{Tables annuelles du Sénat} pour l’année 1931 se présente sous la forme d’un in-octavo relié, composé de plusieurs centaines de pages. La typographie, sobre et régulière, reprend les conventions du \emph{Journal Officiel} : colonnes étroites, numérotation continue, absence d’ornementation. L’ensemble se divise en sections distinctes, qui reflètent les usages concrets des lecteurs.

\subsection{Index des orateurs}

On y trouve une \textbf{liste alphabétique des sénateurs}, chaque nom suivi de références aux séances où ils sont intervenus. Par exemple :

\begin{quote}
\emph{Tardieu (André)} : interventions p. 312, 457, 892.

\end{quote}
Cet index permet de retracer rapidement la présence et l’activité d’un parlementaire sur une année complète. Pour l’historien, il offre une base sérielle pour mesurer la visibilité des élus et la fréquence de leur participation aux débats.

\subsection{Table des matières thématiques}

La deuxième section regroupe les débats par \textbf{matières} :

\emph{ }Finances publiques* : budget, impôts, emprunts.
\emph{ }Affaires étrangères* : traités, conventions, mandats.
\emph{ }Travail et questions sociales* : assurance chômage, législation ouvrière, retraites.

Chaque entrée renvoie à un numéro de séance du \emph{Journal Officiel}. Ce classement thématique reflète une logique documentaire propre, qui diffère de l’ordre chronologique des séances : il met en valeur la récurrence des thèmes et facilite leur repérage transversal.

\subsection{Références législatives et réglementaires}

Enfin, les tables recensent les \textbf{lois votées et les décrets publiés} pendant l’année, assortis de leur date et de leur numéro. Ce registre, proche d’un répertoire législatif, assure le lien avec le \emph{Bulletin des lois} et, par extension, avec l’ensemble de la législation nationale.

 Analyse

Cet exemple illustre trois dimensions essentielles des tables :

\begin{itemize}
\item Leur \textbf{fonction instrumentale} : elles servent avant tout de guide, destiné à faciliter la recherche d’une information précise dans un corpus immense.
\item Leur \textbf{valeur sémantique} : en proposant une catégorisation (par personnes, thèmes, textes), elles produisent une image de l’activité parlementaire qui n’est pas neutre, mais orientée par le mode d’indexation.
\item Leur \textbf{rôle intercorpus} : en mettant en relation débats, interventions et textes normatifs, elles constituent un point de jonction entre la parole parlementaire et le droit promulgué.
\end{itemize}





%%%%%%%%%%% PARTIE 2
\part{L'enjeu des données structurées : des sources à la base de données}


\chapter{Une histoire par les données}

Les \emph{Tables Annuelles} du Sénat, comme on vient de le voir, contiennent une véritable mine d'informations pour établir une analyse de l'activité parlementaire. Ces \emph{Tables}, accessibles sur Gallica, avec le jeu des renvois et des index, sont de véritables bases de données de papier numérisées. Pour récupérer les informations du \emph{Journal Officiel} de façon automatisée, c'est-à-dire sans reproduire à la main l'ensemble, il faut penser à une chaîne de traitement qui part de ces sources numériques, sous format image, pour pouvoir en capturer l'information. Il s'agit ici de voir comment construire un protocole d’extraction cohérent, en tenant compte de la matérialité des documents eux-mêmes, sous leur forme \enquote{analogique}; mais aussi sous leur forme \enquote{numérique}. Dans ce chapitre, il s'agira de répondre aux problématiques technique de cette traduction des sources numérisées -- c'est-à-dire sous format image -- au texte. Ceci imposant de donner un contexte préalable de cette \enquote{mise en données} des sources historiques, laquelle est inhérente à la disponibilité de corpus numérisés par les politiques de valorisation des fonds des institutions patrimoniales. [Section 1: Datafication des corpus : \enquote{numériser}]

Ensuite, premier problème : comment travailler à partir d'une image numérique ? Certes, la représentation photographique et numérique d'un document est lisible pour un oeil humain; mais du point de vue informationnel, ces images ne sont que des paquets de pixels. Ces pixels ne sont pas, évidemment, les lettres elles-mêmes. Ils sont la traduction sur l'écran de trains d'informations binaires qui, sans le bon décodage, pourrait vouloir dire tout autre chose. Le premier enjeu pour un travail de capture de l'information est de transformer cette matière matricielle en information textuelle sur laquelle on peut appliquer des traitements. Le texte se présente comme pré-requis pour établir des chaînes de traitement de capture informationnelle. Ce passage de l'image au texte numérique est en fait techniquement une prérogative des tâches de \emph{reconnaissance optique des caractères} -- ou \enquote{OCR} (Optical Character Recognition). Elle butte également sur des problématiques de détection de la mise en page, laquelle fonde un ordre de lecture -- et donc un agencement du sens des phrases qu'il faut considérer. [Section 2 : de l'image au texte]

Deuxième problème : une fois ce texte numérique obtenu, comment capturer l'information sémantique qui est présente ? Comment l'ordinateur peut comprendre que tel ensemble des caractères alphanumériques correspond en fait à un sénateur de la Troisième République ? On peut trouver, dans le document, des motifs qui signalent une entité (par exemple, un sénateur inaugure chaque paragraphe). Cette approche comme on va le voir, est basée sur la reconnaissance de motifs typographiques. Elle est cependant fragile et dépendante de la qualité de l'OCR -- voire des erreurs humaines présentes dans le document d'origine. Elle suppose aussi une forme de connaissance \emph{a priori} synthétique de la représentation de l'information dans le document. Ainsi peut-on se tourner vers des approche extractive (les Bert) ou bien les approches génératives qui \enquote{lisent} le texte et restitue l'information comprise et permettent de contourner le problème des exceptions qui forment le corps des documents. Cette chaîne de travail de capture est tourné vers un format de cette information exploitable par l'ordinateur. La chaîne de traitement commence donc avec l'image, passe par le texte et des méthodes de capture de l'information sémantique qu'elle contient, pour aboutir à une information structurée. L'enjeu n'est pas simple car chaque étape reporte les marges d'erreur des précédentes. [Section 3]

Dans ce chapitre, il s'agira ainsi de dessiner le contexte technique et institutionnel de cette datafication des données en vue de leur traitement -- et notamment avec les nouvelles opportunités des grands modèles de langage.

\section{Datafication des corpus : \emph{numériser}}

\subsection{En \enquote{mode texte}}

En 1971, un étudiant, reproduisait sur un ordinateur \emph{Xerox} la \emph{Déclaration d'indépendance des Etats-Unis}, en caractères alphanumériques \textbf{ASCII}. Il s'agissait de Michel Hart, fondateur du \textbf{Projet Gutenberg} qui se donnait pour tâche de reproduire et diffuser bénévolement sur le réseau internet des oeuvres littéraires du domaine public. La Bible, les oeuvres de Shakespeare, quelques autres de Lewis Carroll ou de James M. Barrie seront notamment reproduites. Ce travail de \enquote{numérisation} est en fait un travail laborieux : chacune des lettres de chaque livre sera tapée à la main, les unes après les autres. En 1990, de façon contemporaine à la jeunesse du Web, le projet prend un nouvel essor et bénéficie d'une collaboration internationale : les collections s'élèvent à environ 1000 livres en 1997; 4000 livres en 2001; et $15000$ livres en 2005 [Marie Lebert]. Entre le livre et la version numérique, il n'y a pas d'image : juste le travail de transcription manuel des caractères. C'est une numérisation des livres \enquote{en mode texte} [Bermès, 30-33] l'information textuelle seule est reproduite, cela destructurant l'objet livre. Avec cette reproduction en caractères alphanumériques, la structure physique du livre -- sa mise en page -- est perdue; mais on peut en revanche rechercher un mot et retrouver un passage plus aisément. 

Le texte numérique ne se définit pas seulement comme une reproduction électronique du texte imprimé, mais comme une transformation de l’information en une suite de signes codés. Concrètement, chaque caractère est représenté par une valeur numérique, selon un système de codage -- ainsi tel que l’\textbf{ASCII} (American Standard Code for Information Interchange) ou, plus récemment, l’\textbf{Unicode}, qui attribue à chaque lettre, chiffre ou symbole une séquence binaire, une suite de \emph{bits}, c’est-à-dire de 0 et de 1. Ce passage de l’écriture alphabétique à la codification binaire permet au texte d’être manipulé comme une donnée discrète : il devient possible de rechercher automatiquement un mot, de compter des occurrences, de structurer des chaînes de caractères.

Cette démarche d'encodage de l'information, qui ne concerne pas ici proprement l’historien, est exemplaire au regard des méthodes de \emph{numérisation} des documents textuels en ce sens qu’elle traduit une forme analogique — physique ou continue — en une forme numérique, discrète. L’opération de transcription manuelle, caractère par caractère, est ici comparable à celle d’un dépouillement systématique sur archives papier : il s’agit de saisir l’information contenue dans les sources dans un dispositif tabulaire, par exemple un tableur [Claire Lemercier, Claire Zalc]. La similarité n’est toutefois que d’ordre opératoire. Dans le cas de Michel Hart et du Projet Gutenberg, la répétition du texte littéraire reste relativement linéaire et vise une reproduction intégrale, sans problématisation des sources. À l’inverse, la transcription historienne suppose une enquête critique : sélectionner, structurer, et souvent synthétiser des données pour les « mettre en table », c’est-à-dire les rendre comparables et cumulables. Cette dimension opératoire peut être qualifiée, avec Simondon, de travail de \emph{transduction technique} : un processus par lequel l’information passe d’un support et d’un régime de signification à un autre, selon des contraintes à la fois matérielles et intellectuelles. Dans le cas des historiens, ce passage implique un véritable travail d’individuation des données : découper des flux documentaires continus en unités discrètes (noms, dates, professions, événements), qui ne préexistent pas à l’opération de transcription mais sont construites par elle. La numérisation est une opération configuratrice où la saisie manuelle se fait l’instrument d’un changement de régime technique du support de l'information. Du côté des historiens, ce travail de transduction informationnelle, plus sophistiqué que la transcription littéraire, peut être comparé au travail de terrain du sociologue ou de l'éthnographe, en ce sens qu'elle suscite justement des questions et reconfigure les valuations de l'enquête [Dewey; Claire Lemercier, Claire Zalc].

Le travail de saisie manuelle n’est évidemment pas une nouveauté introduite par l’ordinateur. Bien avant l’ère numérique, les historiens s’y adonnaient déjà. Ainsi, à la fin des années 1940, Pierre Chaunu recopiait à la main, sur papier, les données issues des archives et des ouvrages nécessaires à sa thèse, afin de les ordonner et de les exploiter systématiquement [Bertrand Müller]. Aujourd’hui encore, malgré l’apparition d’outils de transcription automatique [voir section 2], cette pratique demeure courante : toutes les sources ne sont pas disponibles en version numérique, et le chercheur, tout comme l’étudiant ou le généalogiste, peut être amené à relever lui-même les informations qui l’intéressent, directement en salle d’archives ou lors du dépouillement de fonds imprimés.

Le mode opératoire de la saisie manuelle constitue en ce sens un exemple singulier, puisqu’il s’oppose radicalement à la logique de la numérisation photographique, dite « en mode image » [Bermès]. Cette opposition met en évidence deux conceptions distinctes de la numérisation : d’un côté, la transcription textuelle, qui construit les données par un travail de sélection et de structuration ; de l’autre, la reproduction visuelle, qui se limite à conserver la matérialité de l’objet. L’histoire des pratiques documentaires, tant individuelles qu’institutionnelles, témoigne de cette tension durable entre deux paradigmes concurrents [Bermès, 29].

\subsection{En \enquote{mode image}}

[Développer en mode image et les politiques institutionnelles : Gallica, Google Book autour de 2005. Salmi] À l’opposé du mode texte, la numérisation en mode image repose sur la capture photographique ou le scan des documents, cherchant à restituer leur matérialité visuelle. Le texte, les blancs, les marges, la typographie, les ornements, tout est \enquote{figé} dans une matrice de pixels. Cette approche a été privilégiée par les grandes politiques de numérisation institutionnelles à partir des années 1990, avec l’émergence de programmes comme Gallica (BnF, 1997) ou Google Books (lancé officiellement en 2004). Ces projets partagent l’ambition d’une mise à disposition massive du patrimoine imprimé, mais ils diffèrent dans leurs logiques : Gallica, qui s’inscrit dans une mission de service public, avec un accent sur la fidélité documentaire, la conservation et l’interopérabilité avec d’autres bibliothèques numériques; ou encore Google Books, de son côté, qui met en avant la puissance de l’indexation, cherchant à rendre \enquote{trouvable} le contenu des livres plutôt qu’à restituer leur intégrité en tant qu’objets. L’essor du mode image a profondément transformé le rapport aux corpus : il permet d’accéder à la forme originale du document, mais rend l’information brute peu exploitable sans traitements complémentaires (OCR, segmentation, structuration). Contrairement au mode texte de Gutenberg, qui sacrifiait la matérialité au profit de la lisibilité, le mode image fige la matérialité mais laisse l’utilisateur face à des images muettes.

Bien sûr, avec les microfilms voire encore avec les méthodes photographiques anciennes, reproduire un document par le moyen de la photographie n'est pas nouveau. Le fac-similé n'a pas ni été inventé par Google ou Gallica. Ce n'est cependant pas seulement le problème de la reproduction textuelle, photographique ou non dont il est question; mais bien de la diffusion et à l'accès à l'information capturée. La \emph{numérisation} de documents, en mode texte ou en mode image, comprennent historiquement les dispositifs techniques terminaux qui peuvent recevoir et reconstituer l'information. Ainsi le système technique de la numérisation qui est autant l'affaire de discrétisation de l'information des documents que de leur diffusion -- à l'instar du dispositif PLAO ou tout simplement du Web. Il est alors intéressant de relier la notion de numérisation à celle de \emph{datafication} pour souligner l'associativité du travail de mise en données à un milieu technique [Latour, Simondon] et social -- c'est-à-dire ici institutionnel.

\subsection{Réseaux et datafication : sphère technique, sphère sociale}

La \emph{datafication} est le processus qui vise à quantifier un phénomène de sorte qu'il soit calculable et analysable [Frédéric Clavert]. Elle considère le passage du continu au discret; la calculabité des phénomènes comme une prérogative du \enquote{numérique}. Cette mise en données \enquote{insiste sur la notion de processus} et \enquote{se définit par les choix opérés par les organismes qui y procèdent}, cela impliquant \enquote{les critères d'inclusion [de] corpus à numériser}; l'élaboration de métadonnées descriptives situées, lesquelles ont un impact sur leur découvrabilité puisque les moteurs de recherche s'y appuient [Frédéric Clavert, 123].

La datafication ne consiste pas simplement à transformer un document papier en fichier numérique : c’est un processus complexe qui combine numérisation, structuration et mise en base de données. En histoire, cela signifie passer du papier (via OCR ou HTR) à un texte exploitable, puis à des entités ou métadonnées normalisées (XML-TEI, bases relationnelles). Ce processus, qui permet de rendre les corpus calculables et interopérables, ouvre certes des perspectives considérables — recherche plein texte, analyses sérielles, croisement de sources — mais il engage aussi une série de choix méthodologiques et institutionnels qui, étant donné leur lourdeur produisent leurs propres biais.

Ces choix se jouent d’abord dans la sélection des corpus à numériser : les bibliothèques privilégient souvent les imprimés homogènes ou les fonds les plus demandés, laissant de côté des séries manuscrites plus fragiles ou moins visibles. Comme l’a montré le projet TIME-US, cette logique favorise la présence de sources officielles ou institutionnelles (journaux ouvriers imprimés, presse syndicale) et relègue dans l’ombre des archives plus marginales (pétitions manuscrites, traces de travail informel). La datafication amplifie ainsi certains silences archivistiques : ce qui n’a pas été consigné, ou ce qui est difficile à transcrire automatiquement, reste hors champ.

S’y ajoutent les biais techniques. L’OCR ou la HTR sont rarement neutres : leur performance varie selon l’état du document, la langue, l’alphabet ou la typographie. Dans TIME-US, il a fallu corriger manuellement des milliers d’occurrences et entraîner des modèles spécifiques pour le français pré-moderne. Par ailleurs, la normalisation des données (dates, métiers, entités) tend à gommer des variations significatives et risque de projeter des catégories anachroniques. De ce point de vue, la datafication ne produit pas des « données brutes », mais bien des données construites, filtrées par des choix techniques.

Enfin, la datafication est aussi un processus social : elle reflète et prolonge les hiérarchies documentaires héritées. Les corpus numérisés surreprésentent souvent les groupes dominants (élites, institutions, employeurs), au détriment des voix minoritaires. Le danger est alors de tomber dans un effet d’« Eldorado numérique » : l’historien travaille sur ce qui est disponible, non sur ce qui est historiquement pertinent. La question devient dès lors : comment documenter ces biais et construire la confiance dans des données issues de systèmes techniques ?

\subsection{Un standard à la croisée de la sphère technique et la sphère sociale : IIIF}

C’est dans ce contexte qu’émergent des standards comme le IIIF (International Image Interoperability Framework), né dans les années 2010 sous l’impulsion d’un consortium de grandes bibliothèques (BnF, British Library, Stanford, etc.). IIIF répond à une double exigence, à savoir : uniformiser l’accès aux images numérisées en définissant des API permettant de zoomer, annoter, partager et intégrer les images dans des environnements divers et favoriser l’interopérabilité entre institutions en permettant qu’un même document numérisé à Londres, Paris ou New York puisse être consulté, manipulé et enrichi dans une interface commune.

IIIF illustre bien l’évolution de la numérisation vers la \emph{datafication} : il ne s’agit plus seulement de stocker et montrer des images, mais de les intégrer dans un écosystème où elles deviennent manipulables, annotables, recombinables. Un manuscrit, une affiche ou un numéro du \emph{Journal Officiel} numérisé n’est plus seulement une image : c’est une ressource ouverte, potentiellement enrichie par des métadonnées, des annotations collaboratives ou des algorithmes d’analyse.

\subsection{\enquote{Le goût de l'archive à l'ère numérique}}

La réflexion sur la datafication des corpus ne peut être dissociée du « goût de l’archive » à l’ère numérique. Le terme, mobilisé par le collectif [retrouver les noms exacts], met en évidence un paradoxe : alors que la numérisation promet une accessibilité inédite aux archives, elle en modifie profondément l’expérience sensible. L’historien ne se trouve plus confronté à des boîtes, des liasses ou des volumes, mais à des corpus massifs, fragmentés, médiés par des interfaces, des moteurs de recherche ou des API.

Ce déplacement fait écho à un débat ancien dans l’historiographie française. Dès les années 1960, Chaunu ou Le Roy Ladurie, dans le sillage de l’École des Annales, avaient déjà revendiqué l’importance des archives sérielles — registres de baptêmes, testaments, cadastres, minutes notariales — au détriment des sources narratives ou « événementielles » jugées plus séduisantes. Ils affirmaient que l’histoire pouvait (et devait) se nourrir de ces « archives grises », répétitives, sans attrait esthétique ni émotionnel, mais capables, une fois cumulées et quantifiées, de révéler des structures profondes (démographiques, sociales, économiques). Autrement dit, l’absence de « goût » de ces archives constituait paradoxalement leur force heuristique.

L’ère numérique radicalise ce basculement : ce qui, dans les années 1970, nécessitait des dépouillements manuels interminables et le recours à l’informatique balbutiante, peut désormais être automatisé et amplifié à une échelle inédite. La datafication prolonge l’intuition des Annales en rendant ces fonds sériels interrogeables et manipulables à grande vitesse. Mais elle modifie aussi l’expérience sensible : le « goût » ne réside plus dans l’objet matériel de l’archive, mais dans la découverte de motifs et de régularités rendus visibles par des visualisations, des bases de données ou des modèles.

Ainsi, l’historien se trouve à nouveau confronté à un paradoxe : les archives sérielles, longtemps délaissées pour leur monotonie, acquièrent une nouvelle attractivité dans l’espace numérique, mais au prix d’un changement de régime du sensible. Le risque est alors de réduire l’archive à son seul potentiel calculable. Le défi, aujourd’hui comme hier, est de concilier la puissance cumulative de ces données avec la vigilance critique nécessaire à l’interprétation de leurs conditions de production.

\section{De l'image au texte : la reconnaissance optique de caractère (OCR)}

Paradigme de l'image numérisée  ==> OCR. Présentation de la dimension technique de l'image numérique matricielle. Information riche et compliquée. C'est le cas de Gallica et spécifiquement du corpus du JO. Problématique de l'extraction repose sur des stratégies d'obtention du texte, à partir duquel on pourra effectuer des traitements extracttifs

Comme on l'a vu dans le chapitre précédent, les \emph{Tables Annuelles} du Sénat sont disponibles sur Gallica. D'un point de vue technique, ces \emph{Tables} sont des documents numérisés, c'est-à-dire des images dont on aura discrétisé l'information. Une image numérique est un tableau -- une \emph{matrice} -- d'une largeur et d'une hauteur données, comportant alors $largeur \times hauteur$ pixels, pixels qui encode l'information colorimétrique sur trois vecteurs : le paramètre $rouge$, le paramètre $vert$, et le paramètre $bleu$. La combinaison de ces trois paramètres, selon les règles de la synthèse colorimétrique additive, permettent de restituer, pour chaque pixel, l'ensemble des couleurs du spectre visible.

\section{Données brutes, données structurées : quelques enjeux de l'interopérabilité.}

Le format .txt, JSON, CSV, XSLX. Envisager

\chapter{Du texte à la donnée structurée : capturer la sémantique}

\section{Approche à motifs explicites : les ReGex}

Une première approche naïve d'extraction de l'information du texte : les regex. Puissants, rapides. Mais rigide et implique de connaître à l'avance la forme de ce qu'on cherche, ce qui n'est pas trivial ! Il faut aussi partir du principe que l'on a pas une connaissance synthétique a priori de l'information. Il y a toujours un \enquote{hic}. Fragile face au bruit ocr, aux fautes typographiques inattendues; et avoir une regex plus souple, c'est aussi prendre le risque de capter du bruit.

\begin{quote}La recherche floue Un moyen de diluer la rigidité des motifs; mais ne permet que de trouver ce que l'on connaît à l'avance. Dans un optique d'extraction massive, on veut tout sortir automatiquement.\end{quote}
\begin{quote}\end{quote}
\begin{quote}> Automates finis !\end{quote}

La contrainte forte des regex Intéressant à coupler avec d'autres approches plus souple comme on le verra.

\begin{itemize}
\item \textbf{Patrons linguistiques} (grammaires, dépendances syntaxiques)
\end{itemize}

\begin{itemize}
\item \textbf{Listes de référence / gazetteers}
\end{itemize}

\begin{itemize}
\item \textbf{Règles de post-traitement}  
\end{itemize}
  => Avantage : explicable, prévisible  
  => Limite : peu robustes aux variations inattendues

\section{Approches extractives : l'approche Bert (one-to-one)}

L'approche du surlignage 1 to 1.

\textbf{Principe} : le modèle apprend à repérer les entités dans un texte via des annotations.

\begin{itemize}
\item \textbf{Modèles supervisés classiques} : CRF, SVM, MaxEnt
\end{itemize}

\begin{itemize}
\item \textbf{Neuraux séquentiels} : BiLSTM-CRF, CNN-LSTM
\end{itemize}

\begin{itemize}
\item \textbf{Transformers extractifs} : BERT, RoBERTa, CamemBERT en mode NER  
\end{itemize}
  => Avantage : généralise mieux, bonne précision  
  => Limite : nécessite des données annotées et un entraînement

\section{Approches génératives : les LLMs}

\textbf{Principe} : le modèle produit directement le résultat structuré à partir du texte, sur la base d’une consigne en langage naturel.

\begin{itemize}
\item \textbf{LLMs} (GPT, Claude, Mistral) en extraction via prompt
\end{itemize}

\begin{itemize}
\item \textbf{Fine-tuning génératif} (T5, GPT-4 en mode extraction JSON)  
\end{itemize}
  => Avantage : très flexible, pas besoin de jeu d’entraînement spécialisé  
  => Limite : variabilité, hallucinations, besoin de validation

\section{Approches hybrides}

\textbf{Principe} : combiner plusieurs catégories dans un flux de traitement.

\begin{itemize}
\item Exemple : Gazetteer pour repérer des entités connues + BERT pour les autres + Regex pour les formats normés + validation humaine  
\end{itemize}
  => Avantage : maximiser précision et rappel  
  => Limite : complexité d’intégration

\chapter{Données et FAIRness : de la valuation et de l'évaluation}

La dimension \enquote{personnelle} des données, la FAIRNESS ==> implique d'expliciter la \enquote{sitaution} des données, de leurs valuations et de leur qualité (évaluation. Nécessité d'évaluer. 

« Most literary scholars would no more simply use the “results” \emph{of a fellow scholar than they would use her toothbrush} » (Responses to Moretti, p. 4). 5




%%%%%%%%% PARTIE 3
\part{Expérimenter et évaluer pour comprendre : une démarche historienne outillée}

%\section{Introduction}
\label{sec:intro}
The growing use of artificial intelligence by historians \cite{clavert2024histoire} is multiplying the possibilities for producing historical datasets. The advent of large language models (LLMs) is further changing the landscape, especially for processing textual data corpora, with a proliferation of uses and experiments in the humanities and social sciences\footnote{The "DH@LLM: Grands modèles de langage et humanités numériques" conference program, held in Paris in July 2025, is a good illustration of this: \href{https://www.crihn.org/nouvelles/2025/01/16/colloque-dhllm-grands-modeles-de-langage-et-humanites-numeriques-sorbonne-universite/}{https://www.crihn.org/nouvelles/2025/01/16/colloque-dhllm-grands-modeles-de-langage-et-humanites-numeriques-sorbonne-universite/}}. Zero-shot LLMs are capable of performing a wide range of tasks without the need for task-specific examples or fine-tuning \cite{kojima2022large,wei2022emergent,zhao2023survey} and have demonstrated their ability to carry out many time-consuming tasks in historical research, such as transcription \cite{humphries2024unlocking}, information extraction \cite{knutsen2024alimenter}, or annotation \cite{yuan2025leveraging}. 

%In the context of historical data extraction, a key issue is obtaining structured outputs. While structured generation -- which constrains a Large Language Model to directly produce information in a predefined format like JSON -- is a way to achieve this, a structured output can also be obtained through post-processing of free-form text generation. Regardless of the method, structured output enables the preservation of \joseph{the link to?} the original document and facilitates return to the source, while still allowing for subsequent processing — such as, for instance, integration into a database.

The use of large language models (LLMs) opens new perspectives for extracting structured data \cite{liu2024structured} from historical documents. In the context of historical data extraction, a central challenge lies in obtaining structured outputs. One approach is structured generation, which constrains a large language model to directly produce information in a predefined format such as JSON. Alternatively, structure can be imposed through post-processing of free-form text outputs. Regardless of the approach, producing structured data enables traceability back to the original document and facilitates source verification. It also supports downstream uses, such as integration into a database or further computational analysis.

%\joseph{Est-ce que le problème n'est pas de dire : 1. comment on passe du texte brute au CSV (pour simplifier) et 2. comment on évalue ce que ça donne ?}

%\marie{Si ça vous convient à tous les deux, je trouve ça mieux.}
%\joel{Oui, ça me convient !} \marie{Ci-dessous, une proposition, en ayant utilisé ce qu'a dit Joseph. J'ai aussi utilisé ce qu'il a dit par Teams.}

Two fundamental issues still remain: (1) how to move from raw text to an exploitable structured representation, such as a table or CSV file ; and (2) how to assess the quality and reliability of the extracted data. This article addresses both aspects through a concrete case study: the extraction of structured information from the 1931 \textit{Tables nominatives} or \textit{Tables des noms} of the French Senate (index of senatorial activity ordered by name). We explore a lightly constrained generation approach using an LLM and propose a method to represent the target data, guide the extraction process, and evaluate system performance. Beyond this specific case, the study aims to contribute to broader reflections on the feasibility and limitations of generative models for structuring historical data.

The \textit{Tables des noms} of the French Senate was published during the French Third Republic (1870–1940)\footnote{These tables are part of the \textit{Tables annuelles} (yearly activity index), which can be consulted on the digital library of the French national library (\textit{BnF}): \href{https://gallica.bnf.fr/ark:/12148/cb371291967/date.item}{https://gallica.bnf.fr/ark:/12148/cb371291967/date.item}.}.
Within the broader documentary ecosystem of the \textit{Journal Officiel} ---~which seeks to reconstruct parliamentary activity and its legal or regulatory outcomes in France~---, the Senate's \textit{Tables nominatives} offer a concise and systematic record of senators' interventions during public sessions.
These indexes were designed to accompany the transcription of debates\footnote{The complete transcriptions of Senate debates can be consulted via Gallica: \href{https://gallica.bnf.fr/ark:/12148/cb34363182v/date}{https://gallica.bnf.fr/ark:/12148/cb34363182v/date}.} and to facilitate their consultation.
Manually compiled once a year, they recorded each intervention by senators or members of the government who spoke during the sessions, the subject of their speech, and the corresponding page number.
%
While these tables were particularly useful at a time when full-text search in digitized parliamentary debates was not possible, they still hold significant value for historians today.
Systematically extracting data from them would make it possible to track parliamentary activity over the long term, quantify the interventions of specific senators affiliated with particular political movements, or support the cross-validation of named entities extracted from the debates themselves.
%
Our objective is to extract structured data from these \textit{Tables}; for our initial experiments, we focus on a single \textit{Table nominative}, namely that of 1931.
The early 1930s marked the beginning of the decline of French parliamentarism, culminating in the fall of the Third Republic in 1940 \cite{morel2024parlement}. Analyzing the 1931 \textit{Table} allows us to lay the groundwork for a broader study that will extend across the entire decade, with the aim of capturing the parliamentary activity of the Senate and, subsequently, of the Chamber of Deputies.


%\textbf{Problem:} \textit{[bon là, la 1ère phrase, j'ai un peu riffé... Je ne sais pas si c'est vrai]} Yet, the evaluation of these outputs remains dominated by symbolic “exact match” metrics: Exact Match, Precision/Recall/F-score at the node or parent-child pair level. These metrics are blind to semantic proximity: a predicted label that differs by just one level in the hierarchy or by a lexical variant is penalized as heavily as a completely unrelated prediction, which distorts the measurement of actual performance.

%\joseph{Partie trop technique pour l'intro selon moi, à déplacer dans la partie "Evaluation Framework"}

%\marie{Dans ce cas, il faut supprimer la Research question et enchaîner direct ensuite sur Case study}

%\textbf{Hypothesis:} Optimal Transport (OT) provides a mathematical framework to measure the minimal cost of transforming one distribution into another: here, the distribution of labels generated by the LLM versus the ground truth. By assigning a cost proportional to the semantic/hierarchical distance between labels, OT softens the penalty: the closer the prediction, the lower the cost. The originality of this case study lies in the combination of OT + structured output, which is rarely explored for document indexing tasks. 

%\textbf{Research question:} To what extent does Optimal Transport offer a more fine-grained, human-aligned evaluation than classical metrics when assessing the structured output of an LLM on a document indexing task?

%\textbf{Case study:} Senate tables. Brief presentation of the source and its usefulness.


After reviewing existing approaches to structured data extraction and evaluation (Section~\ref{sec:related-work}), we present three main contributions.

\chapter{L'outil Corpusense : une chaîne de traitement pour les sources historiques}

L’outil \emph{Corpusense} du projet Mezanno a pour ambition de proposer aux chercheurs et chercheuses en sciences humaines et sociales un dispositif dédié à l’exploitation de corpus d’archives sérielles. Là où la partie précédente du mémoire a montré la diversité des enjeux liés à la numérisation et à la mise à disposition de ces sources — qualité imparfaite de l’OCR, absence de structuration exploitable dans les textes bruts, masse de données qui rend toute saisie manuelle irréaliste — \emph{Corpusense} vise à apporter une réponse technique réaliste. \enquote{Réaliste}, car il s’appuie pour cela sur des briques logicielles déjà disponibles et rapides à mettre en place, qu’il agence en une chaîne cohérente correspondant au schème opératoire transcriptif défini précédemment, allant de la source numérisée à la donnée structurée.

Le fonctionnement de \emph{Corpusense} repose principalement sur l’usage d’\textbf{API}, c’est-à-dire d’interfaces de programmation qui permettent à des logiciels hétérogènes de communiquer entre eux en suivant un protocole défini. Les API fonctionnent comme des guichets d'information qui délivrent, selon les requêtes, des données demandées, par exemple une série d'images et leurs métadonnées dans le cas de \textbf{IIIF}. Plutôt que de réimplémenter des modules complexes, l’outil tire parti de services spécialisés en les appelant directement via leurs API. Trois d’entre elles structurent le dispositif : IIIF, donc, qui est le standard largement adopté par les institutions patrimoniales, qui permet de charger et manipuler des images numérisées de manière normalisée. La constitution de corpus se fait donc à partir des dépôts d'archives numérisées, en général par des institutions patrimoniales. Il faut compter également une \enquote{APIsation} du moteur \textbf{Pero OCR} par l'EPITA; qui fournit comme on l'a vu une reconnaissance optique des caractères adaptée aux corpus historiques et multilingues. Et, enfin, l’API de \textbf{Mistral}, un modèle de langage, mobilisée pour transformer les textes OCRisés en sorties structurées adaptées aux besoins des chercheurs (voir \ref{fig:corpusense}). Ainsi, il s'agit de reconstituer une chaîne opératoire, composé de différentes briques agencées et constituées en outil.

\begin{figure}[htbp]
\centering
\includegraphics[width=\linewidth]{corpusense.png}
\caption{}
\label{fig:corpusense}
\end{figure}

L'outil motive donc une double orientation. D’une part, il repose sur une chaîne de traitement unifiée, pensée pour transformer des documents d’archives sous forme image en données structurées, interopérables et prêtes à être analysées. D’autre part, il se veut flexible et accessible, de façon à laisser aux chercheurs SHS une autonomie réelle dans la conduite de leurs travaux. Concrètement, une application Web comme \emph{Corpusense} permet de constituer des corpus documentaires à partir de dépôts variés d’archives numérisées et d’y appliquer, sans compétences techniques avancées, des traitements qui aboutissent à des données exploitables.

\section{Une instance de pipeline \enquote{classique}}

L’objectif de Corpusense veut fournir une infrastructure technique robuste pour solidariser les différents rouages (à savoir la constitution des corpus, OCR, extraction) \ref{fig:pipieline_mezz}.

\begin{figure}[htbp]
\centering
\includegraphics[width=\linewidth]{pipieline_mezz.png}
\caption{}
\label{fig:pipieline_mezz}
\end{figure}

La chaîne de traitement se déploie en plusieurs étapes successives. Tout d'abord, la sélection et organisation des sources. Le point de départ réside dans la constitution du corpus. Celui-ci peut provenir de fonds institutionnels (par exemple Gallica, la BnF ou des archives universitaires), ou de collections numérisées indépendantes. Les documents, le plus souvent disponibles sous forme d’images, sont alors recensés et organisés dans un format exploitable. Dans ce contexte, \emph{Corpusense} s’appuie sur le protocole IIIF, largement adopté dans le domaine patrimonial. IIIF permet non seulement d’accéder aux images numérisées de manière normalisée, mais aussi de les manipuler (zoomer, rogner, annoter) et de les intégrer de façon homogène, quelle que soit l’institution d’origine. Ce recours à un standard interopérable assure la portabilité des corpus et facilite leur exploitation au-delà du cadre spécifique de ce projet. 

Ensuite, la transcription par OCR. La deuxième étape consiste à convertir ces images en texte grâce à un moteur de reconnaissance optique de caractères (OCR). Corpusense s’appuie principalement sur le moteur PERO OCR. 

Enfin, la sortie structurée (avec l'API Mistral), laquelle produit un document en JSON, qui peut d'ailleurs être converti aisément en format CSV.

Le choix du format JSON pour représenter les données structurées ne relève pas seulement de la contrainte imposée par l’API Mistral même si cette disponibilité oriente un tel choix. Ce format présente en fait plusieurs avantages décisifs dans le cadre d’un outil comme \emph{Corpusense}. D’un point de vue technique, JSON est un standard commun pour l’échange de données : léger, lisible par l’humain, directement exploitable par la plupart des langages de programmation et facilement convertible en d’autres formats, qu’il s’agisse de tables ou de bases de données relationnelles ou documentaires. Cette plasticité garantit la réutilisation des résultats, quel que soit l’environnement de recherche dans lequel ils sont ensuite mobilisés. L’intérêt du JSON réside dans l'implémentation de la dialectique clé/valeur, laquelle se prête particulièrement bien à des tâches \enquote{indexatoires}, davantage convenante à la logique tabulaire, plus rigide. Là où un tableau, par exemple au format CSV, oblige à « aplatir » l’information, JSON permet de conserver les relations entre entités (par exemple entre un intervenant et les différentes pages où il est cité) et d’accueillir des variations de granularité sans perdre la cohérence du tout. Par exemple, chaque sénateur peut intervenir un certain nombre de fois : en JSON, chaque référence d'intervention est un élément manipulable, distinct des autres; et l'ensemble de ces références est une liste de taille variable. Dans un tableau, on pourrait soit agréger ces différentes références dans une unique colonne séparés avec un séparateur arbitraire \ref{fig:comparaison} -- impliquant alors de parser \emph{a posteriori} ces séries de nombres -- ou bien, mais c'est ici une option assez malheureuse, de constituer autant de colonnes que d'interventions. La séparation des valeurs en JSON fait partie de sa grammaire quand, côté tables, la séparation de valeurs numériques est en fait une chaînes de caractères qu'il faut \enquote{spliter} pour retomber sur une certaine modularité.

\begin{figure}[htbp]
\centering
\includegraphics[width=\linewidth]{comparaison.png}
\caption{Comparaison}
\label{fig:comparaison}
\end{figure}

Il est ainsi possible de représenter des cas simples ou complexes au sein d’un même corpus, ce qui correspond mieux à la réalité hétérogène des archives numérisées. A ce stade, on remarque une difficulté pour l'historien qui mobilise des données structurées. Dans une démarche outillée, par exemple avec l'outil \emph{Corpusense}, elle résiderait peut-être moins dans la capacité à programmer qu'à envisager la forme des données, qu'à modéliser un problème ou un ensemble de faits. Cet exercice n’est pas trivial : veut-on décrire des entités et leurs attributs, suivre leur activité dans le temps, ou cartographier leurs relations ? Chacun de ces choix renvoie à des modèles de données distincts, et donc à des manières différentes de faire parler les sources. L'outil ne fait pas disparaître les problématiques de modélisation des données. On retrouve ici la notion de \emph{valuation} au sens deweyien : les moyens de l’enquête — ici, la forme des données et les dispositifs techniques qui les produisent — dépendent des fins poursuivies, mais ces fins elles-mêmes ne sont jamais figées. Elles peuvent être révisées, ajustées ou enrichies au fil des expérimentations, en fonction de ce que les données rendent possible ou non. En ce sens, modéliser les données revient parfois à réactiver, au présent, des gestes interprétatifs analogues à ceux que Collingwood décrivait sous le terme de \emph{reenactment} : l’historien ne fait pas que collecter des informations, il rejoue l’acte de pensée, reconstruit les problèmes tels qu’ils se posaient aux acteurs du passé, mais à travers des médiations techniques. L’historien n’a donc pas affaire à une simple « conversion numérique » de sa pratique, mais bien à une reconfiguration de ses schèmes interprétatifs sous l’effet des opérations techniques. Dans cette perspective, parler de « numérisation du métier d’historien » \footcite[][]{poublanc} ne désigne pas seulement une facilitation instrumentale par les outils numériques : cela renvoie à l’intégration de nouveaux schèmes techniques dans l’enquête elle-même, qui orientent la manière de modéliser les mots et les faits. L’historien ne saurait être l'aliéné des dispositifs techniques — au sens où Simondon entend l’aliénation comme l’usage aveugle d’outils méconnus — mais participe au contraire à leur individuation, en les inscrivant consciemment dans son milieu de recherche et dans ses gestes interprétatifs. 

De plus, cette solidarisation des différentes briques techniques — de cette médiation technique pour constituer des questions et des réponses — ne fait pas disparaître la question de l’évaluation des données produites. Au contraire, elle la rend plus pressante. Car exploiter scientifiquement des données issues d’un assemblage \emph{abstrait} de techniques \emph{concrètes} — pour reprendre le vocabulaire de Gilbert Simondon — suppose de fonder une confiance raisonnée : confiance dans les outils choisis, dans la cohérence de la chaîne opératoire, mais aussi dans la capacité des chercheurs à expliciter les conditions de production des données qui alimentent leur analyse. Dans le cas d’une pipeline comme \emph{Corpusense}, on a affaire à un tel assemblage abstrait : une juxtaposition de fonctions spécialisées (OCR, segmentation, structuration), qui ne forment pas encore un objet intégré mais dont la coopération doit être évaluée comme un tout. Ce qui permet de fonder cette confiance est donc l’\emph{évaluation} de la chaîne de traitement, c’est-à-dire à la fois la performance de chaque module et la cohérence globale de l’ensemble.

\section{Le travail sur Corpusense}

Au moment du stage, \emph{Corpusense} était encore en cours de développement; toutes les fonctionnalités n'étaient pas encore disponibles. Je l’ai donc utilisé principalement pour la constitution du corpus, l’OCRisation et le téléchargement du texte issu de la \emph{Table nominale} de 1931. En revanche, pour la génération de la sortie structurée à l’aide de l’API Mistral, je travaillais directement depuis mon ordinateur, en effectuant les appels manuellement. Cette fonctionnalité est aujourd’hui intégrée dans \emph{Corpusense}, mais elle ne l’était pas encore au moment de mes expérimentations.

Ce décalage n’affecte pas la validité des travaux réalisés : les modules sollicités -- notamment Mistral pour la production JSON -- sont identiques, seul le mode d’appel diffère. Les expérimentations sur la structuration des données ont donc été conduites à partir du texte brut, en dehors de l’application, mais elles prolongent directement le schème opératoire mis en place par \emph{Corpusense}.

\section{Ateliers à l'EHESS et à la BnF}

Avant de passer à l’évaluation proprement dite, arrêtons-nous sur un cas concret qui illustre bien les difficultés rencontrées. Lors d’un atelier mené à l'EHESS par les développeurs de \emph{Corpusense}, une chercheuse — sans formation particulière en informatique — a expérimenté l’outil sur un corpus de la Quatrième République. En une demi-heure, elle a pu obtenir un jeu de données massif, comprenant plus de 1200 entités liées à l’activité parlementaire, plus précisemment sur la production de documents parlementaires. L’exercice montre à quel point la chaîne de traitement peut être efficace et accessible : un travail qui aurait pris des semaines en dépouillement manuel est désormais réalisable en un temps réduit. 

Cependant cette réussite apparente masque plusieurs écueils. La chercheuse, tâche peu facile oblige, a eu du mal à définir précisément son modèle de données : quelles entités retenir ? quelles relations considérer comme pertinentes ? quelle granularité adopter ? Faut-il penser en terme d'acteurs ? De documents produits ? Il a fallu essayer différents modèles pour obtenir un résultat satisfaisant ; redéfinir clairement ce que l'on tenait à savoir sur la période étudiée. On pourrait, en reprenant le mot de Leroy Ladurie, l'historien a tout intérêt à être \emph{designer} -- plutôt d'ailleurs que \enquote{programmeur} -- car l'enjeu est d'être capable de modéliser des données. De plus, même si la structuration a fonctionné, la question centrale reste ouverte : les données ainsi produites sont-elles fiables ? Dans quelle mesure peut-on leur faire confiance pour alimenter une enquête historique, et non seulement une démonstration technique ? A l'issue de cet atelier, qui visait avant tout à expérimenter l'outil qu'à produire de véritables données pour une question de recherche déterminée, le besoin de savoir si les données était fiables était urgent : ainsi, par ce cas exemplaire qui ne fait qu'illustrer une problématique qui a fondé les tenants et aboutissants du stage, la grande question de l'\emph{évaluation} et de la confiance que l'on peut porter aux données.

\chapter{La sortie structurée via LLM appliquée à la Table des Noms du Sénat : une approche empirique}

La pertinence scientifique des analyses repose sur l'évaluation qui ne se réduit pas à un simple contrôle technique. Elle engage une véritable réflexion méthodologique. Dans le cadre du stage, cette réflexion méthodologique s'est adossée à l’expérimentation sur la façon d'évaluer des données, générées par la pipeline. Evaluation et expérimention peut sembler antinomiques. Par évaluation, on entend une dimension scientifique, protocolaire. La notion de \enquote{créativité} que motive au fond celle d'expérimentation et sa dialectique de l'essai-erreur, ne semble pas de mise. Pourtant, la question de la métrique à laquelle s'adosse l'évaluation ne va pas de soi : car il faut évaluer à la fois l'indexation du contenu par le système technique; mais également sa structure de façon conjointe. Il faut également pouvoir apprécier les abscences, les hallucinations; éventuellement, les différentes façons d'OCRiser le texte, tester si un OCR réputé parfait conduit à des résultats finalement très proches de données très imparfaites. De plus, l'évaluation s'adosse à des \emph{vérités terrain} qui, comme on l'a vu pour l'OCR, forment un domaine de référence sur laquelle quantifier les écarts avec la génération du LLM. Ces vérités terrain, qui permettent de fonder une analyse permettant d'émuler l'objectivité, sont pourtant le fruit de décisions qui ne sont pas évidentes : comment nommer ses métadonnées, quelle structure -- ou \emph{schéma} -- conviendrait le mieux à nos données ? La modélisation, comme on vient de le voir également est une véritable affaire de design. Il y a donc un arsenal de paramètres à prendre en compte. Pour arrêter un protocole d'évaluation, il faut dégager des critères de façon empirique -- et donc expérimenter et réduire les paramètres selon ce qu'il semble le plus pertinent. Le mot de Gaston Bachelard selon lequel les instruments seraient de la \enquote{théorie réifiée} est particulièrement approprié à notre cas, car les moyens pour mesurer la qualité des données produites par la pipeline dépend de nos valuations. Cette évaluation prend sens dans un processus itératif où techniques et de valeurs amendées par l'expérimentation et par les objectifs fixés par une question de recherche.

Bien évidemment, cette expérimentation-évaluation s'adosse, non pas à une problématique technique pure, mais à une question de recherche en histoire -- bien qu'elle soit un motif pour guider l'exploration documentaire et computationnelle --, à savoir : cartographier l'activité parlementaire pour l'année 1931, conformément au \emph{reeanctement} collingwoodien suggéré en première partie de ce mémoire sur la qualité du débat parlementaire. Cette motivation historienne, en quatre mois seulement, ne saurait être mené de bout en bout. L'enjeu est donc de maîtriser et évaluer le protocole d'extraction; de tester sa complexité et vérifier qu'il est satisfaisant à ce stade, avant de mener des projets plus ambitieux.

\section{Expérimentations}

\subsection{Prise en main intuitive du problème de la génération de données}

L’expérimentation menée s’est organisée autour de deux volets complémentaires : d’une part, la génération de données à partir des \emph{Tables} parlementaires proprement dite; et, d’autre part, leur évaluation à l’aune d’une vérité terrain soigneusement construite.

Dans un premier temps, le travail a consisté en une phase d’exploration technique visant à s’approprier la tâche d’évaluation des sorties structurées produites par un modèle génératif. L’objectif principal était de vérifier la faisabilité d’une extraction fiable des informations essentielles : les noms des intervenants au Sénat (qu’il s’agisse de sénateurs ou de ministres interpellés) et les dates de leurs interventions.

Ces dates ne figurent pas explicitement dans les \emph{Tables} ; elles doivent être déduites par le biais des références de page, lesquelles constituent des indicateurs temporels indirects. En d’autres termes, la pagination continue du \emph{Journal Officiel} rend possible un fléchage des interventions dans le temps, en reliant chaque entrée des tables à la séance correspondante comme on le verra en temps venu.

Pour produire les données structurées, plusieurs pistes ont été explorées. La plus immédiate a été de recourir à des prompts « naturels » en utilisant directement l'interface Web de Mistral, laissant au modèle une certaine liberté interprétative dans sa réponse. 

\begin{figure}[htbp]
\centering
\includegraphics[width=\linewidth]{mistral.png}
\caption{Illustration du résultat d'un prompt avec Mistral (en simulant le modèle disponible en avril 2025) où je soumets un extrait des Tables OCRisées. Comme on le voit, c'est un résultat qui mélange une liste de sénateurs et politesses qui sont ici inconvenantes à une extraction sémantique systématique.}
\label{fig:mistral}
\end{figure}

L'extraction d'une réponse dans l'environnement d'un chat résiste à l'extraction systématique, notamment parce qu'elle est entourée de politesses qui se surajoutent à l'information  qui nous intéresse -- ici la liste des sénateurs \ref{fig:mistral} . Mais également parce que cette réponse est émise dans un environnement -- en ligne -- qu'il est difficile à intégrer dans une pipeline. Les appels à l'API Mistral permettent donc à la fois d'interroger un LLM comme on l'a fait, mais depuis un script qui peut récupérer les données, les sauvegarder en local, d'appliquer divers traitements.

Ce qui nous permet désormais de justifier l’usage de Mistral : l’appel à l’API était gratuit, sans restrictions, et supportait la génération par sortie structurée -- c’est-à-dire la capacité de contraindre le modèle à suivre un schéma formel représentant nos données. Autrement dit, on peut utiliser, sans aucune installation ou dépense, un LLM pouvant être \enquote{contraint} à formuler du JSON -- et rien que du JSON -- par un schéma formels afin de produire des objets comparables. De fait, ce schéma \ref{fig:modelepy} était produit avec la librairie Pydantic qui permet de convertir des classes Python qui représentent le schéma de données, en modèle JSON.

\begin{figure}[htbp]
\centering
\includegraphics[width=\linewidth]{modelepy.png}
\caption{A gauche, l'implémentation en Pydantic via Python d'un des premiers modèles de données imaginé pour la génération guidée (à droite).}
\label{fig:modelepy}
\end{figure}

Mais un schéma, à lui seul, ne suffit pas : il doit être combiné à un prompt qui guide le modèle. Le tout premier essai reposait sur des instructions très simples :

\begin{quote}
« Extrayez les informations du texte fourni. Je veux la liste des noms et prénoms de toutes les personnes mentionnées (des sénateurs). »  
« Attention au bruit : tout ce qu’il y a dans le texte n’est pas forcément un sénateur. »  
« Voici mon texte : »

\end{quote}
Ces instructions rudimentaires avaient pour objectif principal de vérifier le bon fonctionnement de l’appel à l’API et le retour des données dans le format JSON. En pratique, le pipeline consistait en un script Python qui, à partir du texte OCRisé, soumettait au modèle Mistral un prompt et un schéma, puis récupérait les données produites \footcite[][]{pipeline_mezz}.

Toutefois, dès les premiers tests, des limites sont apparues : les clés choisies dans le modèle, comme \emph{ListeSenateur} ou \emph{Senateur}, étaient trop restrictives. Or, les \emph{Tables nominatives} ne mentionnent pas seulement des sénateurs, mais également des ministres interpellés ou intervenants extérieurs. Un schéma trop strict risquait donc d’induire des omissions. Pour contourner ce problème, il a été décidé d’adopter un vocabulaire plus fonctionnel, centré sur les \emph{intervenants} plutôt que sur une fonction institutionnelle définie de façon explicite. Une fois cette chaîne de traitement permettant de produire des données essayée, il était l'heure de consolider un protocole d'évaluation de cette méthode de génération de métadonnées.

\subsection{Design, prompt et vérité terrain : trouver le bon modèle de données}

Un second volet des expérimentations a porté sur la construction de la vérité terrain, élément indispensable à toute évaluation. En théorie, elle sert de référence absolue pour mesurer les performances du modèle. En pratique, elle résulte d’une série de décisions éditoriales et méthodologiques qui influencent directement les résultats. L’exemple de Louis Barthou, sénateur et ministre de la Guerre en 1931, illustre ce point : fallait-il regrouper toutes ses interventions sous une seule entrée ou distinguer ses apparitions selon ses fonctions \ref{fig:barthou}  ? Le choix de la granularité maximale – séparer chaque occurrence – prévient les confusions d’homonymes, mais complique l’évaluation en raison de la tendance du modèle à fusionner les mentions.

\begin{figure}[htbp]
\centering
\includegraphics[width=\linewidth]{barthou.jpg}
\caption{}
\label{fig:barthou}
\end{figure}

Sur le plan technique, la vérité terrain est lié au prompt et au schéma de données (\ref{fig:modelpy}). Le schéma de données fait office de \enquote{carte} qui vient guider la production textuelle du LLM. Il indique \enquote{où} mettre les bons mots dans les bonnes boîtes, les bons intitulés avec les bonnes clés. Pour éviter de démultiplier les paramètres à tester, le prompt (voir : ANNEXE 31) et le schéma (ANNEXE 32) ont été stabilisés à l'issu d'un processus itératif pour évaluer la granularité adéquat. Dans le cadre de l'évaluation, seul le nom du sénateur et les pages de références ont été retenues. Pour guider au mieux la génération du JSON par le LLM, on ajoute à notre schéma des descriptions pour qu'il puisse être plus attentif pour qu'il puisse plus facilement mettre les données dans les bonnes boîtes \ref{fig:model}.

\begin{figure}[htbp]
\centering
\includegraphics[width=\linewidth]{model.png}
\caption{}
\label{fig:model}
\end{figure}

Le schéma (\ref{fig:model}) joue donc ici le rôle de grammaire opérationnelle : il décrit précisément la structure attendue des sorties (nom, rôle et références paginaires des intervenants), et contraint le modèle à produire un JSON validable. Les premières clés (« Senateur », « ListeSenateur ») ont rapidement montré leurs limites : elles risquaient l'invisibilité d’autres acteurs -- comme les ministres -- et biaisaient l’extraction. La version finale de la vérité terrain (ANNEXE 33) adopte des catégories plus neutres (« Intervenant »), révélant que la modélisation n’est jamais un simple exercice technique mais un choix interprétatif.

Quant aux données produites automatiquement par cette chaîne de traitement stabilisées et appliquées sur l'ensemble des pages des \emph{Tables} du Sénat de 1931 (extraits en annexe : ANNEXE 34) , elles semblent être cohérentes. Mais, à défaut d'évaluation, on ne pas tirer de conclusion décisives.

\subsection{Préparer l'évaluation : comparer, apparier}

L’évaluation des sorties structurées générées par un LLM reposant sur un schéma, un prompt repose donc sur une comparaison systématique entre deux ensembles : la vérité terrain construite manuellement comme référence, et les données produites automatiquement par le modèle. L’objectif est maintenant de mesurer objectivement la fidélité de la génération, d’identifier les omissions, duplications ou erreurs, et d’évaluer la robustesse globale du système. Lors de la phase exploratoire autour de l'évaluation, plusieurs approches méthodologiques et outils de comparaison ont été expérimentés afin de prendre en main les méthodes classiques de mesure de similarité et identifier des indicateurs adaptés à l’évaluation qualitative de données produites par un modèle génératif. 

Comparer une vérité terrain soigneusement construite avec des données générées par un modèle de sortie structurée ne se résume pas à vérifier la présence ou l’absence d’éléments identiques. Plusieurs sources de divergences compliquent l’évaluation : des nœuds peuvent être omis, inventés ou dupliqués ; des erreurs d’OCR altèrent le texte ; des éléments sont regroupés ou mal hiérarchisés dans la structure JSON. La comparaison doit donc dépasser une logique purement textuelle pour capturer ces écarts dans toute leur complexité, tout en restant robuste et interprétable. La comparaison n'est pas un simple \enquote{face à face} entre des données réputées parfaites et des données générées. L'expérimentation a permis de mettre en lumière certaines méthodes -- et d'en rejeter d'autres comme la \emph{Tree Edit Distance}, davantage basée sur la structure des données que leur contenu.

La première étape a consisté à expérimenter la comparaison chaque élément textuel de la vérité terrain avec chaque élément généré, ce qui conduit à aplatir temporairement la structure JSON pour se concentrer sur le contenu. La distance de Levenshtein a été privilégiée, car elle mesure simplement le coût minimal d’édition (insertion, suppression, substitution) entre deux chaînes de caractères ; elle est donc adaptée pour évaluer des données issues d’un pipeline OCR + LLM, où les erreurs sont souvent lexicales plutôt que conceptuelles. Concrètement, ces distances ont été organisées en une \emph{matrice de similarité} (\ref{fig:matricesim}), dans laquelle chaque cellule représente le coût de transformation d’un élément de la vérité terrain en un élément généré. 

\begin{figure}[htbp]
\centering
\includegraphics[width=\linewidth]{matricesim.png}
\caption{Matrice de similarité généré avec Python. Les points clairs représentent des distances faibles, c'est-à-dire des mots très similaires.}
\label{fig:matricesim}
\end{figure}

Ce dispositif visuel met en évidence des phénomènes difficiles à saisir autrement : comme les décalages -- des zones sombres signalent des omissions ou des décalages d’alignement, le modèle ayant « perdu le fil » avant de se resynchroniser -- et les regroupements -- des correspondances multiples sur une même ligne indiquent que plusieurs éléments ont été agrégés dans une seule entrée.

\begin{figure}[htbp]
\centering
\includegraphics[width=\linewidth]{decalage.png}
\caption{Zoom sur la matrice de similarité : un décalage de la diagonale peut signifier une omission.}
\label{fig:decalage}
\end{figure}

L’observation de ces matrices ont d'ailleurs conduit, à la marge, à une réflexion plus conceptuelle. En effet, ces matrices ne forment pas toujours un espace métrique au sens strict, car des propriétés comme l’inégalité triangulaire peuvent être bafouées -- autrement dit, l'ensemble des distances peuvent dessiner un paysage impossible où il n'y a pas de cohérence avec les distances. Pourtant, on observe des zones de cohérence locale qui respectent suffisamment ces propriétés pour fonctionner comme des espaces métriques restreints. Là où il y a émergence de propriétés métriques, il y a l'hypothèse d'une correspondance entre la vérité terrain et les données générées. Deux ensembles identiques comparés forment un espace métrique : on pourrait alors calculer \enquote{l'effort} d'une matricité de similarité pour devenir un véritable espace métrique. Encore une fois, il s'agit d'une réflexion marginale qui n'a pas été développée ensuite. Cependant, cela permet de mettre en avant certains garde-fous dans le cadre de l'évaluation, à savoir le respect de certaines propriétés mathématiques liées à la mesure.

Ensuite reste l'étape de la comparaison proprement dite. Comment mettre correspondance les bonnes paires ? Comment comparer toutes les entrées de la vérité terrain avec celles générées ? Si la matrice de similarité représente toutes les relations entre les vérités terrain et les données générées, c'est une cartographie relationnelle qui ne nous permet pas d'apparier les ensembles en vue d'évaluer la qualité des prédictions; elle ne dit pas comment assembler nos entrées. De même, à cause des desynchronisation, il faut pouvoir aller comparer des éléments qui ne sont pas nécessairement \enquote{en-face}, c'est-à-dire qui ne partage pas le même rang.

Pour ce travail de mise en correspondance, intervient le transport optimal. Introduit par Gaspard Monge en 1781 -- dans son \emph{Mémoire sur la théorie des déblais et remblais} -- et reformulé par l'économiste soviétique Leonid Kantorovitch en 1942, le transport optimal vise à trouver la correspondance la plus économique entre deux ensembles : ici, les éléments de la vérité terrain et ceux générés. Plutôt que de tester toutes les permutations possibles, dont le nombre croît de manière factorielle, le transport optimal représente le problème sous forme d’une matrice de coût -- qui ici peut nous être donnée à partir de la matrice de similarité -- et cherche à minimiser la somme des coûts d’appariement. Cette méthode permet donc d’obtenir un alignement global optimal, même lorsque les ensembles ont des tailles différentes; de détecter les correspondances « justifiées » au sens du coût minimal, tout en identifiant les zones où la correspondance est faible ou artificielle; de dépasser une logique purement séquentielle pour analyser la distribution globale des erreurs et omissions. Notre coût, ici, ce sont les \enquote{coûts d'édition} autrement dit nos distances de Levenstein, exprimée par la matrice de similarité.

Cette approche est intéressante pour documenter l’état des sorties générées : elle ne prétend pas résoudre les problèmes d’ordre ou de sémantique, par exemple, apparier « sénateur » et « parlementaire » demanderait des distances sémantiques -- bien que c'est une piste envisageable avec les \emph{embeddings} dans le cas d'une analyse thématique des interventions parlementaires. En revanche, elle fournit une mesure robuste de dissemblance structurelle que nous offre justement le paysage des distances de Levenstein.En pratique, les vérifications manuelles réalisées sur de petits ensembles de données ont confirmé que les appariements identifiaient correctement les correspondances : chaque élément de la vérité terrain était relié à son équivalent prédit par le modèle. Ces tests ont permis d’esquisser un protocole d’évaluation fiable, en ancrant l’intuition dans des outils mathématiques et informatiques éprouvés.

Il ne s’agit cependant pas d’une démarche inédite : elle transpose à l’analyse des textes structurés un principe ancien et solide de l’évaluation de segmentation en vision par ordinateur. Comme l’explique Chazalon et Carlinet \footcite[][]{chazalon} à propos de la segmentation panoptique de cartes historiques, il est à la fois possible et pertinent de reformuler l’appariement des régions segmentées — contenus visuels — via un cadre bipartite, d’utiliser des métriques classiques (précision, rappel, F-score) \footcite[][]{chazalon}. Appliquée à nos données textuelles, cette approche inspire l’idée que l’appariement entre vérité terrain et données générées ne repose pas seulement sur le calcul statistique, mais peut aussi être qualifié, visualisé et interprété selon des logiques proches de celles de l’image numérique. En tout cas, une phase de l'expérimentation a consisté à comprendre la nature du transport optimal -- notamment avec les travaux de Gabriel Peyré \footcite[][]{peyre} -- afin d'en saisir les limites -- l'appariement ne regarde pas le sens des mots et ne prend pas compte de l'aspect séquentielle des entrées -- et les avantages -- permet justement de surmonter le problème des oublis et de faire correspondre des données qui ne sont pas \enquote{en face}. Cela impliquant ainsi d'intégrer \emph{a posteriori}, dans les métriques qui rendent compte de la qualité des données générées également la qualité de l'appariement.

Toutefois, ces expérimentations mettent également en évidence la dimension heuristique de cette démarche : les résultats dépendent autant du choix des mesures de similarité (distance de Levenshtein) que de la stratégie d’appariement retenue (transport optimal), ce qui engage une réflexion critique sur le design même de l’évaluation. Ainsi, même si la méthode hérite de protocoles éprouvés de l’imagerie, son adaptation à un contexte textuel demande des choix méthodologiques soigneux et explicités.

\section{Recouper des données pour l'analyse historienne}

Dans une perspective historienne, la capacité à naviguer dans les volumes du \emph{Journal officiel} ne repose pas uniquement sur la qualité des extractions textuelles, mais sur le recoupement systématique des données. L’enjeu est de relier des références issues des tables — ici limitées à des numéros de pages et aux occurences nominales — à des dates précises de publication, condition essentielle pour contextualiser les débats ou décisions mentionnés. Pour ce faire, nous avons exploité les manifests IIIF fournis par l’API Gallica, qui décrivent chaque volume sous forme d’une structure hiérarchisée : chaque page est identifiée, ordonnée et associée à une URL stable. Ce cadre technique permet de reconstituer un tableau complet \enquote{page → date}, en combinant l’ordre des pages du manifeste avec les métadonnées des volumes et les dates présentes sur les pages de titre. Cette opération ne vise pas à créer un modèle sophistiqué de reconnaissance ou d’indexation, mais plutôt à bâtir un pivot de référence déterministe pour toute analyse ultérieure : il devient possible de passer d’une citation brute (une page) à son contexte temporel exact. 

Il est important de souligner qu'à ce stade, cette correspondance se fonde sur les pages extraites par le système et rien n'indique que cette extraction repose sur des données fiables. Si ces données, à défaut d'évaluation, ne sont pas exploitables scientifiquement pour répondre à une question de recherche, il est cependant convenable de vérifier qu'on peut, à partir d'elles, avoir des informations plus intéressantes que des références de pages, comme des informations chronologiques.

\subsection{Des pages aux dates : utilisation de l'API Gallica et des métadonnées des manifestes}

La constitution d’un tableau « page → date » s’appuie sur les services IIIF proposés par Gallica, qui fournissent pour chaque document numérisé un manifest JSON décrivant l’ensemble de ses pages. Ce \emph{manifest} joue le rôle de table des matières numérique : il contient, pour chaque page, son ordre dans le volume, son libellé (numéro de page ou folio) et des identifiants pérennes vers l’image et le contenu OCRisé. La première étape consiste à interroger automatiquement cette ressource -- via des requêtes HTTP en lignes de commande, et notamment avec \enquote{curl} -- pour extraire la séquence complète des pages.

Ensuite, les pages sont indexées : chaque entrée du manifest est associée à son URL lisible et à son numéro de séquence. Cette structure séquentielle est cruciale, car elle fournit un pivot neutre pour relier des références issues d’autres sources (tables imprimées, index, etc.).

La seconde étape exploite les pages de titre qui portent la date d’édition. Ces informations, accessibles via l’API Gallica sont extraites puis assignées à des plages de pages contiguës : ainsi, toutes les pages comprises entre deux dates identifiées héritent de la date correspondante.

\begin{figure}[htbp]
\centering
\includegraphics[width=\linewidth]{bnfjson.png}
\caption{Fragment du tableau de correspondance page/date en .csv}
\label{fig:bnfjson}
\end{figure}

En combinant ces deux sources (manifest + extraction de dates), on construit une table de correspondance exhaustive reliant chaque page d’un volume à sa date exacte de parution. Cette table, exportée en CSV, sert ensuite de base à des analyses historiennes : une référence de page extraite d’un index thématique peut être immédiatement replacée dans son contexte temporel. L’approche, entièrement automatisable, repose sur des standards ouverts (IIIF, JSON, HTTP) et sur des outils simples de parsing et de jointure de données, ce qui garantit sa robustesse et sa réutilisabilité. (ANNEXE 34)

\chapter{Livrables}

\section{Jeux de données pour le protocole d'évaluation de la sortie structurée}

\subsection{Protocole d’évaluation de la sortie structurée}

L’objectif de l’évaluation est de comparer les sorties produites par le LLM -- notées P, pour \emph{prédiction} -- au \emph{ground truth} -- la vérité terrain, notée G. Chaque instance de donnée correspond à une liste d’entrées, chacune constituée d’un nom d’orateur et d’une liste de pages référencées.

Un défi majeur, on l'a vu, tient au fait que le modèle peut produire le bon ensemble d’entrées, mais dans un ordre différent, ou avec de légères variations structurelles. Pour résoudre cela, nous adoptons une stratégie d’alignement flexible inspirée utilisant le transport optimal pour établir une correspondance \emph{one-to-one} entre prédictions et référence. Cette approche neutralise la contrainte d’ordre et tolère des divergences mineures, permettant une évaluation robuste.

L'OCRisation est indépendante et en amont de la comparaison :

\begin{enumerate}
\item Chaque page image est d’abord traitée indépendamment par le moteur OCR PERO pour détecter et transcrire le texte.
\item Compte tenu des difficultés persistantes de segmentation de mise en page, nous produisons trois variantes de transcription par page afin de refléter la variabilité de ce type de pipeline : une OCRisation \enquote{brute}, sans segmentation; une OCRisataion qui repose sur une segmentation manuelle; et une émulation d'une OCRisation réputée parfaite, fait à la main, qui correspond à la vérité terrain.

\end{enumerate}
Dans nos expérimentations, pour des raisons pratiques, chaque page est traitée indépendamment, mais la méthode peut être étendue à des contextes multi-pages.

Bien que les LLMs puissent être incités à générer des sorties structurées, il est essentiel de contraindre leurs générations au format attendu. Parmi les méthodes existantes, la plus efficace repose sur le filtrage des tokens valides au moment de l’inférence à l’aide d’un validateur externe (par exemple un automate fini). Cette capacité est disponible dans certaines API commerciales étendues.

Notre processus d’extraction s’appuie donc sur :

\begin{itemize}
\item un texte prétraité par OCR,
\item un schéma prédéfini (JSON),
\item un prompt naturel,
\item et une API supportant les sorties contraintes.

\end{itemize}
Pour cette étude, nous avons retenu l’API Mistral, avec le modèle \emph{Ministral 8B Instruct v2410}, en raison de ses bonnes performances \emph{zero-shot}, de son coût modéré et de la disponibilité publique de ses poids à des fins de recherche.

\subsection{Vérité terrain, prompt et schéma : un échantillon des Tables}

A l'issue de la démarche d'expérimentation en vue de l'évaluation a été produit le design de vérité terrain, accompagné de son schéma et du prompt, qui sera comparées à l'information extraite par le LLM. La tâche d’extraction repose sur l’identification des entrées individuelles dans les \emph{Tables nominatives} — c’est-à-dire les noms des sénateurs et leurs références de pages. (ANNEXE 33, 34)

Un échantillon aléatoire de cinq pages consécutives a été choisi pour transcription et annotation manuelles. Comme les entrées peuvent s’étendre sur plusieurs pages, certaines se retrouvent tronquées. Dans ces cas, le LLM a été explicitement instruit d’ignorer les éléments incomplets. Ce choix reflète les défis réalistes de l’extraction, où une prise en compte multi-pages (ou en flux continu) serait nécessaire en production, mais dépasse le cadre de cette étude.

L’évaluation porte sur 109 entrées réparties sur cinq pages, chacune testée dans les trois conditions OCR.

\subsection{Prédictions obtenues avec le modèle Mistral 8b avec  l'extraction guidée par schéma}

Pour l’année 1931, les \emph{Tables des noms} du Sénat couvrent 14 pages et comptent environ 300 entrées, chacune correspondant à une intervention en assemblée. Chaque entrée est associée à un orateur et détaille différents types d’actions (demandes d’interpellation, discussions de projets de loi, lecture de rapports de commission, dépôt d’amendements, etc.), accompagnées d’une référence de page renvoyant à la transcription complète de l’intervention.

Ces transcriptions sont publiées dans les \emph{Débats parlementaires} du Sénat. Les tables sont donc liées fonctionnellement aux transcriptions par la pagination. Comme la numérotation des pages est continue sur toute l’année, chaque référence permet de dater précisément l’intervention correspondante.

\section{Une métrique pour l'évaluation des données générées par la sortie stucturée via LLM}

\subsection{Mise en correspondance des prédictions et de la vérité terrain avec le transport optimal}

Pour comparer les entrées prédites et de référence, nous définissons une distance normalisée $d_e(g_i, p_j)$ qui combine deux composantes :

\begin{itemize}
\item \textbf{Nom de l’orateur (texte)} : distance de Ratcliff/Obershelp (basée sur la plus longue sous-chaîne commune), après minuscule et suppression des espaces. La distance normalisée $d_n(g_i, p_j) \in [0,1]$ vaut 0 pour un match exact et 1 pour une dissimilarité totale.

\item \textbf{Pages référencées (ensembles)} : distance Intersection-over-Union (IoU) :

\end{itemize}
$$
d_p(g_i, p_j) = 1 - \frac{|ref\_pages(g_i) \cap ref\_pages(p_j)|}{|ref\_pages(g_i) \cup ref\_pages(p_j)|}
$$

\begin{itemize}
\item \textbf{Distance d’entrée} :

\end{itemize}
$$
d_e(g_i, p_j) = d_n(g_i, p_j) \times d_p(g_i, p_j)
$$

Un appariement \emph{one-to-one} entre prédictions et vérité terrain est alors établi par transport optimal [19], minimisant la distance totale et fournissant une base rigoureuse pour l’évaluation.

\subsection{Limites des métriques classiques : précision, rappel et F1}

Les métriques usuelles (précision, rappel, F1) sont couramment utilisées pour évaluer les tâches d’extraction. Cependant, dans notre protocole, où les entrées sont alignées par transport optimal, elles deviennent trompeuses :

\begin{itemize}
\item l’appariement injectif force une correspondance complète, ce qui maximise artificiellement la précision,
\item le rappel ne reflète pas les entrées manquantes ou ajoutées,
\item la F1 hérite de ces biais et surestime les performances.

\end{itemize}
\subsection{Integrated Matching Quality (IMQ)}

Pour dépasser ces limites, nous exploitons directement la distance $d_e\$ pour définir un score de qualité \$q_i = 1 - d_e(g_i, p_i)$, qui reflète la proximité entre une entrée prédite et sa référence.

Plutôt que de fixer un seuil arbitraire pour décider du « bon » appariement, nous calculons la proportion de correspondances de qualité supérieure à un seuil $t\$, puis intégrons sur tout l’intervalle $[0,1]$ :

$$
IMQ = \int_0^1 F(t) \, dt
$$

où $F(t)\$ est la fraction des correspondances de qualité \$q_i \geq t$.

L’IMQ résume ainsi la qualité globale des appariements, récompensant à la fois leur nombre et leur proximité. Un score de 1 indique un alignement parfait. Cette métrique continue, indépendante de seuils arbitraires, est particulièrement adaptée aux sorties LLM, où de légères divergences sont fréquentes même sous fortes contraintes structurelles.

\subsection{Résultats et analyse}

Nous avons appliqué notre méthode d’appariement sur cinq pages distinctes (109 entrées), chacune traitée indépendamment et présentant des qualités OCR variables. Le tableau ci-dessous présente les résultats pour chaque page OCRisée sans segmentation ni correction, avec les tailles des ensembles de référence et prédits, ainsi que le nombre de correspondances retenues par transport optimal :

| Source  | Précision (biaisée) | Rappel (biaisé) | IMQ    | Entrées de référence | Entrées prédites | Correspondances |
| ------- | ------------------- | --------------- | ------ | -------------------- | ---------------- | --------------- |
| page 02 | 1.0000              | 0.9565          | 0.9059 | 23                   | 22               | 22              |
| page 03 | 1.0000              | 1.0000          | 0.8928 | 25                   | 25               | 25              |
| page 04 | 1.0000              | 1.0000          | 0.9591 | 19                   | 19               | 19              |
| page 05 | 1.0000              | 1.0000          | 0.8636 | 19                   | 19               | 19              |
| page 10 | 1.0000              | 1.0000          | 0.8193 | 23                   | 23               | 23              |

Toutes les pages affichent une précision et un rappel « biaisés » parfaits ; mais comme discuté en section précédente, ces métriques sont limitées car elles découlent directement de l’appariement injectif. Elles ne reflètent pas la qualité réelle des alignements.

L’\textbf{IMQ}, en revanche, fournit une évaluation plus fine, en capturant la distribution des qualités de correspondances. Pour toutes les pages traitées, les scores IMQ restent élevés (entre 0.8193 et 0.9591), montrant une homogénéité forte entre correspondances. L’IMQ évalue donc à la fois la complétude et la proximité sémantico-syntaxique des appariements, jouant un rôle hybride entre rappel qualitatif et précision pondérée.

 Variations entre pages

\emph{ \textbf{Pages 5 et 10} : IMQ plus bas, lié à des incohérences typographiques. De nombreux prénoms n’y sont pas mis entre parenthèses après le nom, contrairement à l’attendu dans le }ground truth* (21 \% des entrées sur la page 5, 39 \% sur la page 10). Cela augmente artificiellement la distance textuelle et dégrade la qualité perçue des correspondances.

\begin{itemize}
\item \textbf{Page 3} : malgré une précision/rappel parfaits, IMQ plus faible (0.8928), dû à des problèmes OCR causés par un pli dans la reliure, générant du bruit visuel.

\item \textbf{Page 2} : IMQ élevé (0.9059) malgré un rappel imparfait. Cela s’explique par un biais d’échantillonnage : le prompt avait été calibré sur cette page, ce qui améliore artificiellement la performance. Toutefois, les résultats solides sur la page 4 (IMQ = 0.9591) confirment la robustesse du dispositif.

\end{itemize}
Un cas particulier : sur la page 2, une personne est mentionnée deux fois (comme sénateur et comme ministre). Le \emph{ground truth} distingue ces deux entrées, tandis que le LLM les fusionne. Cela réduit artificiellement le rappel mais correspond à une rationalisation fonctionnelle du modèle.

 Comparaison avec OCR « parfait »

Lorsque l’on compare avec l’OCR jugé « parfait » (corrigé manuellement), les résultats s’améliorent globalement :

| Source  | Précision (biaisée) | Rappel (biaisé) | IMQ    | Entrées de référence | Entrées prédites | Correspondances |
| ------- | ------------------- | --------------- | ------ | -------------------- | ---------------- | --------------- |
| page 02 | 1.0000              | 1.0000          | 0.9513 | 23                   | 23               | 23              |
| page 03 | 1.0000              | 1.0000          | 0.9430 | 25                   | 25               | 25              |
| page 04 | 1.0000              | 1.0000          | 0.9821 | 19                   | 19               | 19              |
| page 05 | 1.0000              | 1.0000          | 0.8778 | 19                   | 19               | 19              |
| page 10 | 1.0000              | 1.0000          | 0.8966 | 23                   | 23               | 23              |

Dans certains cas, les versions OCR bruitées donnent des résultats paradoxalement meilleurs. Par exemple, les en-têtes courants capturés par l’OCR bruité fournissent un contexte utile pour les entrées tronquées en début de page. Ainsi, sur la page 2, le LLM a correctement reproduit la double mention (sénateur/ministre), alors que l’OCR corrigé ne l’a pas permis.

Cela montre que la performance dépend non seulement du LLM, mais aussi de l’adéquation entre ses comportements et la conception du \emph{ground truth}.

\begin{itemize}
\item Le prompt apparaît comme un paramètre critique : certains écarts ne sont pas liés au modèle, mais aux instructions données.
\item Un schéma de granularité raisonnable, couplé à un prompt générique, permet d’obtenir des résultats fiables sans nécessiter une connaissance « atomique » des spécificités documentaires.
\item L’analyse statistique page par page révèle des indices sur les exceptions structurelles internes aux documents (choix typographiques ou institutionnels), qui peuvent être significatives pour l’historien.

\end{itemize}
\section{Conclusion}

Ce travail a exploré l’utilisation des grands modèles de langage pour la génération de données structurées à partir de sources historiques, à travers une étude de cas centrée sur les \emph{Tables nominatives} du Sénat français de 1931. L’approche — combinant OCR, structuration guidée par schéma et génération contrainte via LLM — a produit des résultats évalués grâce à une métrique plus adaptée, l’\textbf{IMQ}, intégrée dans un protocole d’alignement optimal reliant données de référence et données prédites.

L’introduction de la métrique IMQ s’est révélée essentielle : elle permet d’évaluer la qualité de structuration au-delà des scores classiques de précision/rappel, inadéquats dans ce contexte.

Plusieurs pistes s’ouvrent pour renforcer la robustesse et la généralisation de l’approche :

\begin{itemize}
\item \textbf{Relier plus directement données extraites et questions de recherche} : il s’agit d’assurer que les hypothèses de réponse formulées à partir des données générées restent robustes dans le temps.
\item \textbf{Évaluer le prompt lui-même} : cette étape reste à formaliser pour parvenir à un protocole d’évaluation véritablement complet.
\item \textbf{Repenser la structuration de données} : elle ne doit pas être considérée comme un simple prétraitement neutre, mais comme un choix déterminant pour les analyses historiques possibles.

\end{itemize}
De ce point de vue, le prompt et le schéma de données apparaissent comme des \textbf{méta-paramètres} du système de production de données historiques. Leur génération et leur ajustement doivent être conçus comme faisant partie intégrante de la chaîne de traitement.

Une voie prometteuse consiste à \textbf{systématiser et automatiser ce processus de méta-optimisation}, afin de rendre ces approches reproductibles, transparentes et accessibles à des utilisateurs non spécialistes.

\begin{itemize}
\item Brown et al. → `\footcite[][]{brown}`

\item \footcite[][]{purentimeus}, \footcite[][]{poudra}, \footcite[][]{lamasse}, \footcite[][]{eugenedroitpol} \footcite[][]{saudrais} \footcite[][]{lemesle}, \footcite[][]{bonnard} \footcite[][]{coniez} \footcite[][]{morel} \footcite[][]{prost} \footcite[][]{coniez}

\item \footcite[][]{dirjo} \footcite[][]{jo1931} \footcite[][]{rygiel} \footcite[][]{gardey} \footcite[][]{boutier} \footcite[][]{comete} \footcite[][]{chazalon}\footcite[][]{com}\footcite[][]{comm}

\item Chen et al. → `\footcite[][]{chen}`

\item Clavert & Muller → `\footcite[][]{clavertmuller}`

\item Devlin et al. → `\footcite[][]{devlin}`

\item Finkel, Grenager & Manning → `\footcite[][]{finkelmanning}`

\item Humphries et al. → `\footcite[][]{humphries}`

\item Kirillov et al. → `\footcite[][]{kirillov}`

\item Kišš, Beneš & Hradiš → `\footcite[][]{kisshradis}`

\item Knutsen → `\footcite[][]{knutsen}`

\item Kodym & Hradiš → `\footcite[][]{kodymhradis}`

\item Kohút & Hradiš → `\footcite[][]{kohuthradis}`

\item Kojima et al. → `\footcite[][]{kojima}`

\item Liu et al. → `\footcite[][]{liu}`

\item de Marneffe et al. → `\footcite[][]{marneffe}`

\item Mintz et al. → `\footcite[][]{mintz}`

\item Mistral AI → `\footcite[][]{mistralai}`

\item Morel → `\footcite[][]{morel}`

\item Nadeau & Sekine → `\footcite[][]{nadeausekine}`

\item Peyré & Cuturi → ``

\item Radford et al. (2018) → `\footcite[][]{radford2018}`

\item Radford et al. (2019) → `\footcite[][]{radford2019}`

\item Wei et al. → `\footcite[][]{wei}`

\item Willard & Louf → `\footcite[][]{willardlouf}`

\item Yuan & Sester → `\footcite[][]{yuansester}`

\item Zhang & Shasha → `\footcite[][]{zhangshasha}`

\item Zhao et al. → `\footcite[][]{zhao}`
\end{itemize}




\part{Conclusion}

%%%%%%%%%%%%%%%%%%

\appendix %Des appendices: tables figures, etc

\chapter[Titre court]{Le titre très long de la première annexe}

%\input{fichier.tex}

\newpage{\pagestyle{empty}\cleardoublepage}

%%%%%%%%%%%%%%%%%%

\backmatter % glossaire, index, table des figures, table des matières.. (la bibliographie a déjà été appelée)

%\printindex
%\printglossaries[title=Glossaire]
%\listoftables
%\listoffigures
\tableofcontents
\end{document}
